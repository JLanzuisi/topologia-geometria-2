% Options for packages loaded elsewhere
\PassOptionsToPackage{unicode}{hyperref}
\PassOptionsToPackage{hyphens}{url}
%
\documentclass[
  10pt,
  spanish,
  twocolumn,DIV=18,toc=flat]{scrartcl}
\usepackage{lmodern}
\usepackage{amssymb,amsmath}
\usepackage{ifxetex,ifluatex}
\ifnum 0\ifxetex 1\fi\ifluatex 1\fi=0 % if pdftex
  \usepackage[T1]{fontenc}
  \usepackage[utf8]{inputenc}
  \usepackage{textcomp} % provide euro and other symbols
\else % if luatex or xetex
  \usepackage{unicode-math}
  \defaultfontfeatures{Scale=MatchLowercase}
  \defaultfontfeatures[\rmfamily]{Ligatures=TeX,Scale=1}
\fi
% Use upquote if available, for straight quotes in verbatim environments
\IfFileExists{upquote.sty}{\usepackage{upquote}}{}
\IfFileExists{microtype.sty}{% use microtype if available
  \usepackage[]{microtype}
  \UseMicrotypeSet[protrusion]{basicmath} % disable protrusion for tt fonts
}{}
\makeatletter
\@ifundefined{KOMAClassName}{% if non-KOMA class
  \IfFileExists{parskip.sty}{%
    \usepackage{parskip}
  }{% else
    \setlength{\parindent}{0pt}
    \setlength{\parskip}{6pt plus 2pt minus 1pt}}
}{% if KOMA class
  \KOMAoptions{parskip=half}}
\makeatother
\usepackage{xcolor}
\IfFileExists{xurl.sty}{\usepackage{xurl}}{} % add URL line breaks if available
\IfFileExists{bookmark.sty}{\usepackage{bookmark}}{\usepackage{hyperref}}
\hypersetup{
  pdftitle={Tarea 3},
  pdfauthor={Jhonny Lanzuisi},
  pdflang={es-ES},
  hidelinks,
  pdfcreator={LaTeX via pandoc}}
\urlstyle{same} % disable monospaced font for URLs
\usepackage{color}
\usepackage{fancyvrb}
\newcommand{\VerbBar}{|}
\newcommand{\VERB}{\Verb[commandchars=\\\{\}]}
\DefineVerbatimEnvironment{Highlighting}{Verbatim}{commandchars=\\\{\}}
% Add ',fontsize=\small' for more characters per line
\newenvironment{Shaded}{}{}
\newcommand{\AlertTok}[1]{\textcolor[rgb]{1.00,0.00,0.00}{\textbf{#1}}}
\newcommand{\AnnotationTok}[1]{\textcolor[rgb]{0.38,0.63,0.69}{\textbf{\textit{#1}}}}
\newcommand{\AttributeTok}[1]{\textcolor[rgb]{0.49,0.56,0.16}{#1}}
\newcommand{\BaseNTok}[1]{\textcolor[rgb]{0.25,0.63,0.44}{#1}}
\newcommand{\BuiltInTok}[1]{#1}
\newcommand{\CharTok}[1]{\textcolor[rgb]{0.25,0.44,0.63}{#1}}
\newcommand{\CommentTok}[1]{\textcolor[rgb]{0.38,0.63,0.69}{\textit{#1}}}
\newcommand{\CommentVarTok}[1]{\textcolor[rgb]{0.38,0.63,0.69}{\textbf{\textit{#1}}}}
\newcommand{\ConstantTok}[1]{\textcolor[rgb]{0.53,0.00,0.00}{#1}}
\newcommand{\ControlFlowTok}[1]{\textcolor[rgb]{0.00,0.44,0.13}{\textbf{#1}}}
\newcommand{\DataTypeTok}[1]{\textcolor[rgb]{0.56,0.13,0.00}{#1}}
\newcommand{\DecValTok}[1]{\textcolor[rgb]{0.25,0.63,0.44}{#1}}
\newcommand{\DocumentationTok}[1]{\textcolor[rgb]{0.73,0.13,0.13}{\textit{#1}}}
\newcommand{\ErrorTok}[1]{\textcolor[rgb]{1.00,0.00,0.00}{\textbf{#1}}}
\newcommand{\ExtensionTok}[1]{#1}
\newcommand{\FloatTok}[1]{\textcolor[rgb]{0.25,0.63,0.44}{#1}}
\newcommand{\FunctionTok}[1]{\textcolor[rgb]{0.02,0.16,0.49}{#1}}
\newcommand{\ImportTok}[1]{#1}
\newcommand{\InformationTok}[1]{\textcolor[rgb]{0.38,0.63,0.69}{\textbf{\textit{#1}}}}
\newcommand{\KeywordTok}[1]{\textcolor[rgb]{0.00,0.44,0.13}{\textbf{#1}}}
\newcommand{\NormalTok}[1]{#1}
\newcommand{\OperatorTok}[1]{\textcolor[rgb]{0.40,0.40,0.40}{#1}}
\newcommand{\OtherTok}[1]{\textcolor[rgb]{0.00,0.44,0.13}{#1}}
\newcommand{\PreprocessorTok}[1]{\textcolor[rgb]{0.74,0.48,0.00}{#1}}
\newcommand{\RegionMarkerTok}[1]{#1}
\newcommand{\SpecialCharTok}[1]{\textcolor[rgb]{0.25,0.44,0.63}{#1}}
\newcommand{\SpecialStringTok}[1]{\textcolor[rgb]{0.73,0.40,0.53}{#1}}
\newcommand{\StringTok}[1]{\textcolor[rgb]{0.25,0.44,0.63}{#1}}
\newcommand{\VariableTok}[1]{\textcolor[rgb]{0.10,0.09,0.49}{#1}}
\newcommand{\VerbatimStringTok}[1]{\textcolor[rgb]{0.25,0.44,0.63}{#1}}
\newcommand{\WarningTok}[1]{\textcolor[rgb]{0.38,0.63,0.69}{\textbf{\textit{#1}}}}
\usepackage{graphicx}
\makeatletter
\def\maxwidth{\ifdim\Gin@nat@width>\linewidth\linewidth\else\Gin@nat@width\fi}
\def\maxheight{\ifdim\Gin@nat@height>\textheight\textheight\else\Gin@nat@height\fi}
\makeatother
% Scale images if necessary, so that they will not overflow the page
% margins by default, and it is still possible to overwrite the defaults
% using explicit options in \includegraphics[width, height, ...]{}
\setkeys{Gin}{width=\maxwidth,height=\maxheight,keepaspectratio}
% Set default figure placement to htbp
\makeatletter
\def\fps@figure{htbp}
\makeatother
\setlength{\emergencystretch}{3em} % prevent overfull lines
\providecommand{\tightlist}{%
  \setlength{\itemsep}{0pt}\setlength{\parskip}{0pt}}
\setcounter{secnumdepth}{-\maxdimen} % remove section numbering
\RecustomVerbatimEnvironment{Highlighting}{Verbatim}{commandchars=\\\{\},fontfamily=mlmr,frame=leftline,numbers=left,numbersep=2.5pt}

\setlength{\fboxsep}{5pt}
\setlength{\columnsep}{20pt}

\setkomafont{title}{\normalfont\sffamily}
\setkomafont{disposition}{\normalfont\sffamily}
\setkomafont{subtitle}{\normalfont\large\sffamily}
\setkomafont{section}{\normalfont\Large\sffamily}
\setkomafont{subsection}{\normalfont\large\sffamily}

\titlehead{Universidad Simón Bolívar\hfill Matemáticas Puras y Aplicadas}
\usepackage{mlmodern}
\ifxetex
  % Load polyglossia as late as possible: uses bidi with RTL langages (e.g. Hebrew, Arabic)
  \usepackage{polyglossia}
  \setmainlanguage[]{spanish}
\else
  \usepackage[shorthands=off,main=spanish]{babel}
\fi

\title{Tarea 3}
\usepackage{etoolbox}
\makeatletter
\providecommand{\subtitle}[1]{% add subtitle to \maketitle
  \apptocmd{\@title}{\par {\large #1 \par}}{}{}
}
\makeatother
\subtitle{Topología y Geometría II}
\author{Jhonny Lanzuisi}
\date{19 de Junio de 2022}

\begin{document}
\maketitle

{
\setcounter{tocdepth}{3}
\tableofcontents
}
\begin{Shaded}
\begin{Highlighting}[]
\CommentTok{\# Configuración de python y gráficos}
\ImportTok{import}\NormalTok{ sympy }\ImportTok{as}\NormalTok{ sym}
\ImportTok{import}\NormalTok{ numpy }\ImportTok{as}\NormalTok{ np}
\ImportTok{import}\NormalTok{ matplotlib.style}
\ImportTok{import}\NormalTok{ matplotlib.pyplot }\ImportTok{as}\NormalTok{ plt}
\ImportTok{import}\NormalTok{ matplotlib}

\NormalTok{matplotlib.style.use(}\StringTok{\textquotesingle{}seaborn\textquotesingle{}}\NormalTok{)}

\NormalTok{matplotlib.rcParams.update(\{}
        \StringTok{\textquotesingle{}figure.autolayout\textquotesingle{}}\NormalTok{: }\VariableTok{True}\NormalTok{,}
\NormalTok{\})}

\KeywordTok{def}\NormalTok{ lp(input\_list):}
    \ControlFlowTok{for}\NormalTok{ i }\KeywordTok{in}\NormalTok{ input\_list:}
        \BuiltInTok{print}\NormalTok{(}\StringTok{\textquotesingle{}$$\textquotesingle{}} \OperatorTok{+}\NormalTok{ sym.latex(i) }\OperatorTok{+} \StringTok{\textquotesingle{}$$\textquotesingle{}}\NormalTok{)}
\end{Highlighting}
\end{Shaded}

\hypertarget{primera-pregunta}{%
\section{Primera pregunta}\label{primera-pregunta}}

Sean \[
    E = \left\{\frac{x^2}{a} + \frac{y^2}{b} + \frac{z^2}{c} = 1\right\}
    \quad
    \text{y}
    \quad
    S^2 = \{x^2 + y^2 + z^2 = 1\}
\] el elipsoide y la esfera, respectivamente.

Queremos hallar una función \(f\colon S^2\to E\). Esto es lo mismo que
decir que \(f\) debe cumplir: \[
    \frac{f_x}{a^2} + \frac{f_y}{b^2} + \frac{f_z}{c^2} = 1.
\]

Si hacemos \(f_x=ax, f_y=by, f_z=cz\) entonces la ecuación anterior se
cumple, tomando en cuenta que \((x,y,z)\in S^2\).

Tenemos entonces la función \(f\) definida por: \[
    f(x,y,z) = (ax, by, cz).
\]

Esta función es claramente continua y diferenciable, pues sus
componentes lo son. Más aún, su inversa \[
    f^{-1}(x,y,z) = \left(\frac{x}{a}, \frac{y}{b}, \frac{z}{c}\right)
\] es igualmente continua y diferenciable siempre que \(a,c,b\) no sea
ninguno cero, lo cual se cumple para todos los puntos del dominio.

Por último, notemos que \(f\) es un mapa lineal y \(\ker f = \{0\}\) lo
que nos dice que \(f\) es biyectiva.

Tenemos entonces que \(f\) es un difeomorfismo de \(S^2\) a \(E\).

\hypertarget{segunda-pregunta}{%
\section{Segunda pregunta}\label{segunda-pregunta}}

El mapa \(A\) es lineal puesto que sus componentes son mapas lineales.
Un mapa lineal entre espacios de dimensión finita siempre es
diferenciable, por lo que \(A\) es diferenciable.

También, \(A^{-1}\) es un mapa lineal puesto que \(A=A^{-1}\).

Por último el \(\ker A = \{0\}\), por lo que \(A\) tiene rango máximo y
es biyectiva.

Se tiene entonces que \(A\) es un difeomorfismo.

\hypertarget{tercera-pregunta}{%
\section{Tercera pregunta}\label{tercera-pregunta}}

Hace falta comprobar las tres propiedades de toda relación de
equivalencia.

\hypertarget{reflexividad}{%
\subsection{Reflexividad}\label{reflexividad}}

Sea \(S\) una superficie regular y consideremos el difeomorfismo dado
por la función identidad. Es claro que la identidad es biyectiva y que
su inversa es diferenciable. Tenemos entonces que \(S\) es difeomorfa a
si misma, que es lo que queríamos demostrar.

\hypertarget{simetruxeda}{%
\subsection{Simetría}\label{simetruxeda}}

Sea \(S_1,S_2\) superficies regulares difeomorfas por una función \(f\).
Entonces por definición \(f^{-1}\) es diferenciable y al ser \(f\) una
biyección \(f^{-1}\) también lo es. Se cumple entonces que \(S_1\) es
difeomorfo a \(S_2\) (por \(f\)) si, y solo si, \(S_2\) es difeomorfo a
\(S_1\) (por \(f^{-1}\)).

\hypertarget{transitividad}{%
\subsection{Transitividad}\label{transitividad}}

Sean \(S_1,S_2,S_3\) superficies regulares. Sean \(f\colon S_1\to S_2\)
y \(g\colon S_2\to S_3\) difeomorfismos. Entonces la composición
\(g\circ f\colon S_1\to S_3\) es un difeomorfismo, pues el hecho de que
\(f,g\) y \(f^{-1},g^{-1}\) sean diferenciables implica que \(g\circ f\)
y su inversa \(f^{-1}\circ g^{-1}\) son diferenciables, además la
biyectividad de \(f,g\) implica la biyectividad de la composición.

\hypertarget{cuarta-pregunta}{%
\section{Cuarta pregunta}\label{cuarta-pregunta}}

Sea \(p_0=(x_0, y_0, z_0)\). Basta con ver que \(\mathrm{d}f_{p_0}\) es
perpendicular al plano tangente, pues la ecuación del plano esta dada
por un vector normal y un punto del plano.

Por la definición del plano tangente existe una curva
\(\alpha:(-\epsilon,\epsilon)\to S\) tal que \(\alpha(0) = p_0\) y
\(\alpha'(0)\) es un vector que esta en el plano tangente.

Notemos que \[
    f\circ\alpha = 0
\] puesto que la imagen de \(f\) para cualquier punto de \(S\) es cero y
el codominio de \(\alpha\) es \(S\).

Podemos entonces, usando la regla de la cadena, calcular
\((f\circ\alpha)'(0)\): \[
    (f\circ\alpha)'(0) = \langle\mathrm{d}f_{\alpha(0)},\alpha'(0)\rangle,
\] pero como \(f\circ\alpha = 0\) la derivada anterior también debe ser
cero.

Se sigue entonces que \(\mathrm{d}f_{p_0}\) es perpendicular a
\(\alpha'(0)\) y por lo tanto al plano tangente.

\hypertarget{quinta-pregunta}{%
\section{Quinta pregunta}\label{quinta-pregunta}}

El dibujo de \(\varphi\) se encuentra en la figura \ref{p1}.

\begin{Shaded}
\begin{Highlighting}[]
\NormalTok{x,y,z,u,v,a }\OperatorTok{=}\NormalTok{ sym.symbols(}\StringTok{\textquotesingle{}x,y,z,u,v,a\textquotesingle{}}\NormalTok{)}

\NormalTok{x }\OperatorTok{=}\NormalTok{ v}\OperatorTok{*}\NormalTok{sym.cos(u)}
\NormalTok{y }\OperatorTok{=}\NormalTok{ v}\OperatorTok{*}\NormalTok{sym.sin(u)}
\NormalTok{z }\OperatorTok{=}\NormalTok{ a}\OperatorTok{*}\NormalTok{u}

\NormalTok{graph }\OperatorTok{=}\NormalTok{ sym.plotting.plot3d\_parametric\_surface(}
\NormalTok{    x,y,z.subs(a,}\DecValTok{1}\NormalTok{),}
\NormalTok{    (u,}\OperatorTok{{-}}\DecValTok{3}\NormalTok{,}\DecValTok{3}\NormalTok{),}
\NormalTok{    (v,}\OperatorTok{{-}}\DecValTok{3}\NormalTok{,}\DecValTok{3}\NormalTok{),}
\NormalTok{    show}\OperatorTok{=}\VariableTok{False}\NormalTok{,}
\NormalTok{)}

\NormalTok{graph.save(}\StringTok{\textquotesingle{}p1.pdf\textquotesingle{}}\NormalTok{)}
\end{Highlighting}
\end{Shaded}

El vector normal viene dado por \[
    N(u,v)
    =
    \frac
        { \varphi_u\wedge\varphi_v }
        { |\varphi_u\wedge\varphi_v| }
    (u,v)
\]

Podemos usar sympy para calcular las derivadas y el producto externo.
Primero \(\varphi_u\):

\begin{Shaded}
\begin{Highlighting}[]
\NormalTok{phiv, phiu }\OperatorTok{=}\NormalTok{ sym.symbols(}\StringTok{\textquotesingle{}phiv, phiu\textquotesingle{}}\NormalTok{)}

\NormalTok{phiu }\OperatorTok{=}\NormalTok{ sym.Matrix([x.diff(u),y.diff(u),z.diff(u)])}
\NormalTok{lp([phiu])}
\end{Highlighting}
\end{Shaded}

\[\left[\begin{matrix}- v \sin{\left(u \right)}\\v \cos{\left(u \right)}\\a\end{matrix}\right]\]

Ahora, \(\varphi_v\):

\begin{Shaded}
\begin{Highlighting}[]
\NormalTok{phiv }\OperatorTok{=}\NormalTok{ sym.Matrix([x.diff(v),y.diff(v),z.diff(v)])}
\NormalTok{lp([phiv])}
\end{Highlighting}
\end{Shaded}

\[\left[\begin{matrix}\cos{\left(u \right)}\\\sin{\left(u \right)}\\0\end{matrix}\right]\]

El vector normal es entonces:

\begin{Shaded}
\begin{Highlighting}[]
\NormalTok{outp }\OperatorTok{=}\NormalTok{ phiu.cross(phiv)}
\NormalTok{outpnorm }\OperatorTok{=}\NormalTok{ sym.sqrt(outp.dot(outp))}

\NormalTok{lp([sym.simplify(outp}\OperatorTok{/}\NormalTok{outpnorm)])}
\end{Highlighting}
\end{Shaded}

\[\left[\begin{matrix}- \frac{a \sin{\left(u \right)}}{\sqrt{a^{2} + v^{2}}}\\\frac{a \cos{\left(u \right)}}{\sqrt{a^{2} + v^{2}}}\\- \frac{v}{\sqrt{a^{2} + v^{2}}}\end{matrix}\right]\]

\begin{Shaded}
\begin{Highlighting}[]
\NormalTok{N }\OperatorTok{=}\NormalTok{ sym.simplify(outp}\OperatorTok{/}\NormalTok{outpnorm)}
\end{Highlighting}
\end{Shaded}

Calculemos ahora la tangente del ángulo que forman el plano tangente
\(T_{\phi(u_0,v)S}\) y el eje \(z\).

La tangente entre dos vectores \(u,v\) esta dada por \[
    \tan(\theta) = \frac{|u\wedge v|}{\langle u,v\rangle},
\] podemos usar la expresión anterior para calcular la tangente buscada,
tomando en cuenta que el ángulo entre el vector normal \(N\) y el eje
\(z\) es complementario al ángulo entre el plano y el eje, la tangente
buscada esta dada por \[
    \tan(\theta+\frac{\pi}{2})
    =
    \frac{\cos(\theta)}{-\sin(\theta)}
    =
    \frac{\langle u,v\rangle}{-|u\wedge v|}
\]

\begin{Shaded}
\begin{Highlighting}[]

\NormalTok{zunit }\OperatorTok{=}\NormalTok{ sym.Matrix([}\DecValTok{0}\NormalTok{,}\DecValTok{0}\NormalTok{,}\DecValTok{1}\NormalTok{])}

\NormalTok{tang }\OperatorTok{=}\NormalTok{ N.dot(N)}\OperatorTok{/{-}}\NormalTok{sym.sqrt(}
\NormalTok{           N.cross(zunit).dot(N.cross(zunit))}
\NormalTok{       )}

\NormalTok{lp([sym.simplify(tang)])}
\end{Highlighting}
\end{Shaded}

\[- \frac{1}{\sqrt{\frac{a^{2}}{a^{2} + v^{2}}}}\]

La expresión anterior se puede simplificar a \[
    - \frac{|v|}{|a|},
\]

Que es proporcional a \(v\), la distancia del punto \(\varphi(u_0,v)\)
al eje \(z\).

\onecolumn
\appendix

\hypertarget{gruxe1ficos}{%
\section{Gráficos}\label{gruxe1ficos}}

\begin{figure}
\includegraphics[width=1\linewidth]{p1} \caption{\label{p1} Grafico $\varphi(u,v)$ para $a=1$}\label{fig:unnamed-chunk-7}
\end{figure}

\end{document}
