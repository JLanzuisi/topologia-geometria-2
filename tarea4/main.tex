% Options for packages loaded elsewhere
\PassOptionsToPackage{unicode}{hyperref}
\PassOptionsToPackage{hyphens}{url}
%
\documentclass[
  10pt,
  spanish,
  toc=flat]{scrartcl}
\usepackage{lmodern}
\usepackage{amssymb,amsmath}
\usepackage{ifxetex,ifluatex}
\ifnum 0\ifxetex 1\fi\ifluatex 1\fi=0 % if pdftex
  \usepackage[T1]{fontenc}
  \usepackage[utf8]{inputenc}
  \usepackage{textcomp} % provide euro and other symbols
\else % if luatex or xetex
  \usepackage{unicode-math}
  \defaultfontfeatures{Scale=MatchLowercase}
  \defaultfontfeatures[\rmfamily]{Ligatures=TeX,Scale=1}
\fi
% Use upquote if available, for straight quotes in verbatim environments
\IfFileExists{upquote.sty}{\usepackage{upquote}}{}
\IfFileExists{microtype.sty}{% use microtype if available
  \usepackage[]{microtype}
  \UseMicrotypeSet[protrusion]{basicmath} % disable protrusion for tt fonts
}{}
\makeatletter
\@ifundefined{KOMAClassName}{% if non-KOMA class
  \IfFileExists{parskip.sty}{%
    \usepackage{parskip}
  }{% else
    \setlength{\parindent}{0pt}
    \setlength{\parskip}{6pt plus 2pt minus 1pt}}
}{% if KOMA class
  \KOMAoptions{parskip=half}}
\makeatother
\usepackage{xcolor}
\IfFileExists{xurl.sty}{\usepackage{xurl}}{} % add URL line breaks if available
\IfFileExists{bookmark.sty}{\usepackage{bookmark}}{\usepackage{hyperref}}
\hypersetup{
  pdftitle={Tarea 4},
  pdfauthor={Jhonny Lanzuisi},
  pdflang={es-ES},
  hidelinks,
  pdfcreator={LaTeX via pandoc}}
\urlstyle{same} % disable monospaced font for URLs
\usepackage{color}
\usepackage{fancyvrb}
\newcommand{\VerbBar}{|}
\newcommand{\VERB}{\Verb[commandchars=\\\{\}]}
\DefineVerbatimEnvironment{Highlighting}{Verbatim}{commandchars=\\\{\}}
% Add ',fontsize=\small' for more characters per line
\newenvironment{Shaded}{}{}
\newcommand{\AlertTok}[1]{\textcolor[rgb]{1.00,0.00,0.00}{\textbf{#1}}}
\newcommand{\AnnotationTok}[1]{\textcolor[rgb]{0.38,0.63,0.69}{\textbf{\textit{#1}}}}
\newcommand{\AttributeTok}[1]{\textcolor[rgb]{0.49,0.56,0.16}{#1}}
\newcommand{\BaseNTok}[1]{\textcolor[rgb]{0.25,0.63,0.44}{#1}}
\newcommand{\BuiltInTok}[1]{#1}
\newcommand{\CharTok}[1]{\textcolor[rgb]{0.25,0.44,0.63}{#1}}
\newcommand{\CommentTok}[1]{\textcolor[rgb]{0.38,0.63,0.69}{\textit{#1}}}
\newcommand{\CommentVarTok}[1]{\textcolor[rgb]{0.38,0.63,0.69}{\textbf{\textit{#1}}}}
\newcommand{\ConstantTok}[1]{\textcolor[rgb]{0.53,0.00,0.00}{#1}}
\newcommand{\ControlFlowTok}[1]{\textcolor[rgb]{0.00,0.44,0.13}{\textbf{#1}}}
\newcommand{\DataTypeTok}[1]{\textcolor[rgb]{0.56,0.13,0.00}{#1}}
\newcommand{\DecValTok}[1]{\textcolor[rgb]{0.25,0.63,0.44}{#1}}
\newcommand{\DocumentationTok}[1]{\textcolor[rgb]{0.73,0.13,0.13}{\textit{#1}}}
\newcommand{\ErrorTok}[1]{\textcolor[rgb]{1.00,0.00,0.00}{\textbf{#1}}}
\newcommand{\ExtensionTok}[1]{#1}
\newcommand{\FloatTok}[1]{\textcolor[rgb]{0.25,0.63,0.44}{#1}}
\newcommand{\FunctionTok}[1]{\textcolor[rgb]{0.02,0.16,0.49}{#1}}
\newcommand{\ImportTok}[1]{#1}
\newcommand{\InformationTok}[1]{\textcolor[rgb]{0.38,0.63,0.69}{\textbf{\textit{#1}}}}
\newcommand{\KeywordTok}[1]{\textcolor[rgb]{0.00,0.44,0.13}{\textbf{#1}}}
\newcommand{\NormalTok}[1]{#1}
\newcommand{\OperatorTok}[1]{\textcolor[rgb]{0.40,0.40,0.40}{#1}}
\newcommand{\OtherTok}[1]{\textcolor[rgb]{0.00,0.44,0.13}{#1}}
\newcommand{\PreprocessorTok}[1]{\textcolor[rgb]{0.74,0.48,0.00}{#1}}
\newcommand{\RegionMarkerTok}[1]{#1}
\newcommand{\SpecialCharTok}[1]{\textcolor[rgb]{0.25,0.44,0.63}{#1}}
\newcommand{\SpecialStringTok}[1]{\textcolor[rgb]{0.73,0.40,0.53}{#1}}
\newcommand{\StringTok}[1]{\textcolor[rgb]{0.25,0.44,0.63}{#1}}
\newcommand{\VariableTok}[1]{\textcolor[rgb]{0.10,0.09,0.49}{#1}}
\newcommand{\VerbatimStringTok}[1]{\textcolor[rgb]{0.25,0.44,0.63}{#1}}
\newcommand{\WarningTok}[1]{\textcolor[rgb]{0.38,0.63,0.69}{\textbf{\textit{#1}}}}
\usepackage{graphicx}
\makeatletter
\def\maxwidth{\ifdim\Gin@nat@width>\linewidth\linewidth\else\Gin@nat@width\fi}
\def\maxheight{\ifdim\Gin@nat@height>\textheight\textheight\else\Gin@nat@height\fi}
\makeatother
% Scale images if necessary, so that they will not overflow the page
% margins by default, and it is still possible to overwrite the defaults
% using explicit options in \includegraphics[width, height, ...]{}
\setkeys{Gin}{width=\maxwidth,height=\maxheight,keepaspectratio}
% Set default figure placement to htbp
\makeatletter
\def\fps@figure{htbp}
\makeatother
\setlength{\emergencystretch}{3em} % prevent overfull lines
\providecommand{\tightlist}{%
  \setlength{\itemsep}{0pt}\setlength{\parskip}{0pt}}
\setcounter{secnumdepth}{-\maxdimen} % remove section numbering
\RecustomVerbatimEnvironment{Highlighting}{Verbatim}{commandchars=\\\{\},fontfamily=mlmr,frame=leftline,numbers=left,numbersep=2.5pt}

\setlength{\fboxsep}{5pt}
\setlength{\columnsep}{20pt}

\setkomafont{title}{\normalfont\sffamily}
\setkomafont{disposition}{\normalfont\sffamily}
\setkomafont{subtitle}{\normalfont\large\sffamily}
\setkomafont{section}{\normalfont\Large\sffamily}
\setkomafont{subsection}{\normalfont\large\sffamily}

\titlehead{Universidad Simón Bolívar\hfill Matemáticas Puras y Aplicadas}
\usepackage{mlmodern}
\ifxetex
  % Load polyglossia as late as possible: uses bidi with RTL langages (e.g. Hebrew, Arabic)
  \usepackage{polyglossia}
  \setmainlanguage[]{spanish}
\else
  \usepackage[shorthands=off,main=spanish]{babel}
\fi

\title{Tarea 4}
\usepackage{etoolbox}
\makeatletter
\providecommand{\subtitle}[1]{% add subtitle to \maketitle
  \apptocmd{\@title}{\par {\large #1 \par}}{}{}
}
\makeatother
\subtitle{Topología y Geometría II}
\author{Jhonny Lanzuisi}
\date{23 de Junio de 2022}

\begin{document}
\maketitle

{
\setcounter{tocdepth}{3}
\tableofcontents
}
\begin{Shaded}
\begin{Highlighting}[]
\ImportTok{import}\NormalTok{ sympy }\ImportTok{as}\NormalTok{ sym}

\KeywordTok{def}\NormalTok{ lp(input\_list):}
    \ControlFlowTok{for}\NormalTok{ i }\KeywordTok{in}\NormalTok{ input\_list:}
        \BuiltInTok{print}\NormalTok{(}\StringTok{\textquotesingle{}$$\textquotesingle{}} \OperatorTok{+}\NormalTok{ sym.latex(i) }\OperatorTok{+} \StringTok{\textquotesingle{}$$\textquotesingle{}}\NormalTok{)}
\end{Highlighting}
\end{Shaded}

\hypertarget{primera-pregunta}{%
\section{Primera pregunta}\label{primera-pregunta}}

Veamos los coeficientes de la primera forma fundamental en un punto
\(p\).

Calculemos primero \(\phi_u\):

\begin{Shaded}
\begin{Highlighting}[]
\NormalTok{a,u,v,b }\OperatorTok{=}\NormalTok{ sym.symbols(}\StringTok{\textquotesingle{}a,u,v,b\textquotesingle{}}\NormalTok{)}

\NormalTok{phi }\OperatorTok{=}\NormalTok{ sym.Matrix([}
\NormalTok{              a}\OperatorTok{*}\NormalTok{u}\OperatorTok{*}\NormalTok{sym.cosh(v),}
\NormalTok{              b}\OperatorTok{*}\NormalTok{u}\OperatorTok{*}\NormalTok{sym.sinh(v),}
\NormalTok{              u}\OperatorTok{**}\DecValTok{2}
\NormalTok{])}

\NormalTok{phi\_u }\OperatorTok{=}\NormalTok{ sym.Derivative(phi, u)}

\NormalTok{lp([phi\_u.doit()])}
\end{Highlighting}
\end{Shaded}

\[\left[\begin{matrix}a \cosh{\left(v \right)}\\b \sinh{\left(v \right)}\\2 u\end{matrix}\right]\]

Luego \(\varphi_v\):

\begin{Shaded}
\begin{Highlighting}[]
\NormalTok{phi\_v }\OperatorTok{=}\NormalTok{ sym.Derivative(phi, v)}

\NormalTok{lp([phi\_v.doit()])}
\end{Highlighting}
\end{Shaded}

\[\left[\begin{matrix}a u \sinh{\left(v \right)}\\b u \cosh{\left(v \right)}\\0\end{matrix}\right]\]

El primer coficiente \(E(u,u) = \langle \varphi_u,\varphi_u\rangle\) es:

\begin{Shaded}
\begin{Highlighting}[]
\NormalTok{E }\OperatorTok{=}\NormalTok{ phi\_u.doit().dot(phi\_u.doit())}

\NormalTok{lp([E])}
\end{Highlighting}
\end{Shaded}

\[a^{2} \cosh^{2}{\left(v \right)} + b^{2} \sinh^{2}{\left(v \right)} + 4 u^{2}\]

El segundo coeficiente \(F(u,v) = \langle \phi_u,\phi_v\rangle\) es:

\begin{Shaded}
\begin{Highlighting}[]
\NormalTok{F }\OperatorTok{=}\NormalTok{ phi\_u.doit().dot(phi\_v.doit())}

\NormalTok{lp([F])}
\end{Highlighting}
\end{Shaded}

\[a^{2} u \sinh{\left(v \right)} \cosh{\left(v \right)} + b^{2} u \sinh{\left(v \right)} \cosh{\left(v \right)}\]

Y el tercer coeficiente \(G(v,v) = \langle\phi_v,\phi_v\rangle\) es:

\begin{Shaded}
\begin{Highlighting}[]
\NormalTok{G }\OperatorTok{=}\NormalTok{ phi\_v.doit().dot(phi\_v.doit())}

\NormalTok{lp([G])}
\end{Highlighting}
\end{Shaded}

\[a^{2} u^{2} \sinh^{2}{\left(v \right)} + b^{2} u^{2} \cosh^{2}{\left(v \right)}\]

Entonces, dado un vector \(\omega\in T_p S\) escrito en la base de
\(T_p S\) inducida por \(\varphi\),
\(\omega = r\varphi_u + s\varphi_v\), \(I(w)\) viene dado por:

\begin{Shaded}
\begin{Highlighting}[]
\NormalTok{omega,r,s }\OperatorTok{=}\NormalTok{ sym.symbols(}\StringTok{\textquotesingle{}omega,r,s\textquotesingle{}}\NormalTok{)}

\NormalTok{omega }\OperatorTok{=}\NormalTok{ r}\OperatorTok{*}\NormalTok{phi\_u.doit() }\OperatorTok{+}\NormalTok{ s}\OperatorTok{*}\NormalTok{phi\_v.doit()}

\NormalTok{lp([omega.dot(omega)])}
\end{Highlighting}
\end{Shaded}

\[4 r^{2} u^{2} + \left(a r \cosh{\left(v \right)} + a s u \sinh{\left(v \right)}\right)^{2} + \left(b r \sinh{\left(v \right)} + b s u \cosh{\left(v \right)}\right)^{2}\]

\hypertarget{segunda-pregunta}{%
\section{Segunda pregunta}\label{segunda-pregunta}}

Veamos los coeficientes de la primera forma fundamental.

Calculemos primero \(\varphi_\rho\):

\begin{Shaded}
\begin{Highlighting}[]
\NormalTok{rho,theta }\OperatorTok{=}\NormalTok{ sym.symbols(}\StringTok{\textquotesingle{}rho,theta\textquotesingle{}}\NormalTok{)}

\NormalTok{phi }\OperatorTok{=}\NormalTok{ sym.Matrix([}
\NormalTok{              rho}\OperatorTok{*}\NormalTok{sym.cos(theta),}
\NormalTok{              rho}\OperatorTok{*}\NormalTok{sym.sin(theta)}
\NormalTok{])}

\NormalTok{phi\_rho }\OperatorTok{=}\NormalTok{ sym.Derivative(phi, rho)}

\NormalTok{lp([phi\_rho.doit()])}
\end{Highlighting}
\end{Shaded}

\[\left[\begin{matrix}\cos{\left(\theta \right)}\\\sin{\left(\theta \right)}\end{matrix}\right]\]

Luego \(\varphi_\theta\):

\begin{Shaded}
\begin{Highlighting}[]
\NormalTok{phi\_theta }\OperatorTok{=}\NormalTok{ sym.Derivative(phi, theta)}

\NormalTok{lp([phi\_theta.doit()])}
\end{Highlighting}
\end{Shaded}

\[\left[\begin{matrix}- \rho \sin{\left(\theta \right)}\\\rho \cos{\left(\theta \right)}\end{matrix}\right]\]

El primer coficiente
\(E(\rho,\rho) = \langle \varphi_\rho,\varphi_\rho\rangle\) es:

\begin{Shaded}
\begin{Highlighting}[]
\NormalTok{E }\OperatorTok{=}\NormalTok{ phi\_rho.doit().dot(phi\_rho.doit())}

\NormalTok{lp([sym.simplify(E)])}
\end{Highlighting}
\end{Shaded}

\[1\]

El segundo coeficiente
\(F(\rho,\theta) = \langle \phi_\rho,\phi_\theta\rangle\) es:

\begin{Shaded}
\begin{Highlighting}[]
\NormalTok{F }\OperatorTok{=}\NormalTok{ phi\_rho.doit().dot(phi\_theta.doit())}

\NormalTok{lp([F])}
\end{Highlighting}
\end{Shaded}

\[0\]

Y el tercer coeficiente
\(G(\theta,\theta) = \langle\phi_\theta,\phi_\theta\rangle\) es:

\begin{Shaded}
\begin{Highlighting}[]
\NormalTok{G }\OperatorTok{=}\NormalTok{ phi\_theta.doit().dot(phi\_theta.doit())}

\NormalTok{lp([sym.simplify(G)])}
\end{Highlighting}
\end{Shaded}

\[\rho^{2}\]

Entonces, dado un vector \(\omega\in T_p S\) escrito en la base de
\(T_p S\) inducida por \(\varphi\),
\(\omega = r\varphi_u + s\varphi_v\), \(I(w)\) viene dado por:

\begin{Shaded}
\begin{Highlighting}[]
\NormalTok{omega }\OperatorTok{=}\NormalTok{ r}\OperatorTok{*}\NormalTok{phi\_rho.doit() }\OperatorTok{+}\NormalTok{ s}\OperatorTok{*}\NormalTok{phi\_theta.doit()}

\NormalTok{lp([sym.simplify(omega.dot(omega))])}
\end{Highlighting}
\end{Shaded}

\[r^{2} + \rho^{2} s^{2}\]

\hypertarget{tercera-pregunta}{%
\section{Tercera pregunta}\label{tercera-pregunta}}

Basta con ver que, al sustituir la expresión para \(\nabla f\) dada, \[
    \langle\nabla f, v\rangle_p = df_p(v)
\] para \(v = \varphi_u,\varphi_v\), puesto que dado un \(v\) cualquiera
podemos escribir \(v = a\varphi_u + b\varphi_v\) y entonces:
\begin{align*}
    \langle\nabla f, v\rangle_p
    &=
    \langle\nabla f, a\varphi_u + b\varphi_v\rangle_p\\
    &=
    a\langle\nabla f, \varphi_u\rangle_p + b\langle\nabla f, \varphi_v\rangle_p.
\end{align*}

Notemos que \(df_p(\varphi_u) = f_u\), por un lado, y que \begin{align*}
    \langle\nabla f, \varphi_u\rangle
    &=
    \left\langle
        \frac{ f_uG - f_vF }{ EG - F^2 } \varphi_u
        + \frac{ f_vG - f_uF }{ EG - F^2 } \varphi_v
        ,
        \varphi_u
    \right\rangle \\
    &=
    \frac{ f_uG - f_vF }{ EG - F^2 } E
    + \frac{ f_vG - f_uF }{ EG - F^2 } F\\
    &=
    \frac{ f_u (EG - F^2) }{EG - F^2}\\
    &=
    f_u\\
    &=
    df_p(\varphi_u).
\end{align*}

Ahora, se puede hacer una verificación similar para \(\varphi_v\):
\begin{align*}
    \langle\nabla f, \varphi_v\rangle
    &=
    \left\langle
        \frac{ f_uG - f_vF }{ EG - F^2 } \varphi_u
        + \frac{ f_vG - f_uF }{ EG - F^2 } \varphi_v
        ,
        \varphi_v
    \right\rangle \\
    &=
    \frac{ f_uG - f_vF }{ EG - F^2 } F
    + \frac{ f_vG - f_uF }{ EG - F^2 } G\\
    &=
    \frac{ f_v (EG - F^2) }{EG - F^2}\\
    &=
    f_v\\
    &=
    df_p(\varphi_v).
\end{align*}

\hypertarget{cuarta-pregunta}{%
\section{Cuarta pregunta}\label{cuarta-pregunta}}

\hypertarget{primera-parte}{%
\subsection{Primera parte}\label{primera-parte}}

Supongamos que \(S_1\) es orientable. Entonces existe una familia de
parametrizaciones \(\{\varphi_\alpha\}\),
\(\varphi_\alpha\colon U_\alpha\to S_1\), que cubre a \(S_1\) tal que si
\(p\in \varphi_\alpha(U_\alpha)\cap\varphi_\beta(U_\beta)\) entonces la
función de cambio de variable \(\varphi_\beta^{-1}\circ\varphi_\alpha\)
es tal que \(\det(d(\varphi_\beta^{-1}\circ\varphi_\alpha)) > 0\).

Consideremos la familia de parametrizaciones de \(S_2\) dada por
\(\{f\circ\varphi_\alpha\}\),
\(f\circ\varphi_\alpha\colon U_\alpha\to S_2\). Entonces si
\(p\in f\circ\varphi_\alpha(U_\alpha)\cap f\circ\varphi_\beta(U_\beta)\)
la función de cambio de coordenadas viene dada por: \[
    (f\circ\varphi_\beta)^{-1}\circ (f\circ\varphi_\alpha)
    =
    \varphi_\beta^{-1}\circ f^{-1}\circ f\circ\varphi_\alpha
    =
    \varphi_\beta^{-1}\circ\varphi_\alpha,
\] y \(\det(d(\varphi_\beta^{-1}\circ\varphi_\alpha)) > 0\) como vimos
antes. Por lo que \(S_2\) es orientable.

Si suponemos ahora que \(S_2\) es orientable. Entonces existe una
familia de parametrizaciones \(\{\varphi_\alpha\}\),
\(\varphi_\alpha\colon U_\alpha\to S_2\), que cubre a \(S_2\) tal que si
\(p\in \varphi_\alpha(U_\alpha)\cap\varphi_\beta(U_\beta)\) entonces la
función de cambio de variable \(\varphi_\beta^{-1}\circ\varphi_\alpha\)
es tal que \(\det(d(\varphi_\beta^{-1}\circ\varphi_\alpha)) > 0\).

Consideremos la familia de parametrizaciones de \(S_1\) dada por
\(\{f^{-1}\circ\varphi_\alpha\}\),
\(f^{-1}\circ\varphi_\alpha\colon U_\alpha\to S_1\). Entonces si
\(p\in f^{-1}\circ\varphi_\alpha(U_\alpha)\cap f^{-1}\circ\varphi_\beta(U_\beta)\)
la función de cambio de coordenadas viene dada por: \[
    (f^{-1}\circ\varphi_\beta)^{-1}\circ (f^{-1}\circ\varphi_\alpha)
    =
    \varphi_\beta^{-1}\circ f\circ f^{-1}\circ\varphi_\alpha
    =
    \varphi_\beta^{-1}\circ\varphi_\alpha,
\] y \(\det(d(\varphi_\beta^{-1}\circ\varphi_\alpha)) > 0\) como vimos
antes. Por lo que \(S_1\) es orientable.

\hypertarget{segunda-parte}{%
\subsection{Segunda parte}\label{segunda-parte}}

La orientación inducida por \(f\) viene dada por la familia
\(\{f\circ\varphi_\alpha\}\),
\(f\circ\varphi_\alpha\colon U_\alpha\to S_2\), del ejercicio anterior.

Como \(A\) es un operador lineal \(dA = A\) y se sigue que \[
    det(dA) = -1,
\] pues \(A\) es el negativo de la matriz identidad. Tenemos entonces
que el operador \(A\) invierte la orientación.

\end{document}
