% Options for packages loaded elsewhere
\PassOptionsToPackage{unicode}{hyperref}
\PassOptionsToPackage{hyphens}{url}
%
\documentclass[
  10pt,
  spanish,
  twocolumn,DIV=18,toc=flat]{scrartcl}
\usepackage{lmodern}
\usepackage{amssymb,amsmath}
\usepackage{ifxetex,ifluatex}
\ifnum 0\ifxetex 1\fi\ifluatex 1\fi=0 % if pdftex
  \usepackage[T1]{fontenc}
  \usepackage[utf8]{inputenc}
  \usepackage{textcomp} % provide euro and other symbols
\else % if luatex or xetex
  \usepackage{unicode-math}
  \defaultfontfeatures{Scale=MatchLowercase}
  \defaultfontfeatures[\rmfamily]{Ligatures=TeX,Scale=1}
\fi
% Use upquote if available, for straight quotes in verbatim environments
\IfFileExists{upquote.sty}{\usepackage{upquote}}{}
\IfFileExists{microtype.sty}{% use microtype if available
  \usepackage[]{microtype}
  \UseMicrotypeSet[protrusion]{basicmath} % disable protrusion for tt fonts
}{}
\makeatletter
\@ifundefined{KOMAClassName}{% if non-KOMA class
  \IfFileExists{parskip.sty}{%
    \usepackage{parskip}
  }{% else
    \setlength{\parindent}{0pt}
    \setlength{\parskip}{6pt plus 2pt minus 1pt}}
}{% if KOMA class
  \KOMAoptions{parskip=half}}
\makeatother
\usepackage{xcolor}
\IfFileExists{xurl.sty}{\usepackage{xurl}}{} % add URL line breaks if available
\IfFileExists{bookmark.sty}{\usepackage{bookmark}}{\usepackage{hyperref}}
\hypersetup{
  pdftitle={Tarea 2},
  pdfauthor={Jhonny Lanzuisi},
  pdflang={es-ES},
  hidelinks,
  pdfcreator={LaTeX via pandoc}}
\urlstyle{same} % disable monospaced font for URLs
\usepackage{color}
\usepackage{fancyvrb}
\newcommand{\VerbBar}{|}
\newcommand{\VERB}{\Verb[commandchars=\\\{\}]}
\DefineVerbatimEnvironment{Highlighting}{Verbatim}{commandchars=\\\{\}}
% Add ',fontsize=\small' for more characters per line
\newenvironment{Shaded}{}{}
\newcommand{\AlertTok}[1]{\textcolor[rgb]{1.00,0.00,0.00}{\textbf{#1}}}
\newcommand{\AnnotationTok}[1]{\textcolor[rgb]{0.38,0.63,0.69}{\textbf{\textit{#1}}}}
\newcommand{\AttributeTok}[1]{\textcolor[rgb]{0.49,0.56,0.16}{#1}}
\newcommand{\BaseNTok}[1]{\textcolor[rgb]{0.25,0.63,0.44}{#1}}
\newcommand{\BuiltInTok}[1]{#1}
\newcommand{\CharTok}[1]{\textcolor[rgb]{0.25,0.44,0.63}{#1}}
\newcommand{\CommentTok}[1]{\textcolor[rgb]{0.38,0.63,0.69}{\textit{#1}}}
\newcommand{\CommentVarTok}[1]{\textcolor[rgb]{0.38,0.63,0.69}{\textbf{\textit{#1}}}}
\newcommand{\ConstantTok}[1]{\textcolor[rgb]{0.53,0.00,0.00}{#1}}
\newcommand{\ControlFlowTok}[1]{\textcolor[rgb]{0.00,0.44,0.13}{\textbf{#1}}}
\newcommand{\DataTypeTok}[1]{\textcolor[rgb]{0.56,0.13,0.00}{#1}}
\newcommand{\DecValTok}[1]{\textcolor[rgb]{0.25,0.63,0.44}{#1}}
\newcommand{\DocumentationTok}[1]{\textcolor[rgb]{0.73,0.13,0.13}{\textit{#1}}}
\newcommand{\ErrorTok}[1]{\textcolor[rgb]{1.00,0.00,0.00}{\textbf{#1}}}
\newcommand{\ExtensionTok}[1]{#1}
\newcommand{\FloatTok}[1]{\textcolor[rgb]{0.25,0.63,0.44}{#1}}
\newcommand{\FunctionTok}[1]{\textcolor[rgb]{0.02,0.16,0.49}{#1}}
\newcommand{\ImportTok}[1]{#1}
\newcommand{\InformationTok}[1]{\textcolor[rgb]{0.38,0.63,0.69}{\textbf{\textit{#1}}}}
\newcommand{\KeywordTok}[1]{\textcolor[rgb]{0.00,0.44,0.13}{\textbf{#1}}}
\newcommand{\NormalTok}[1]{#1}
\newcommand{\OperatorTok}[1]{\textcolor[rgb]{0.40,0.40,0.40}{#1}}
\newcommand{\OtherTok}[1]{\textcolor[rgb]{0.00,0.44,0.13}{#1}}
\newcommand{\PreprocessorTok}[1]{\textcolor[rgb]{0.74,0.48,0.00}{#1}}
\newcommand{\RegionMarkerTok}[1]{#1}
\newcommand{\SpecialCharTok}[1]{\textcolor[rgb]{0.25,0.44,0.63}{#1}}
\newcommand{\SpecialStringTok}[1]{\textcolor[rgb]{0.73,0.40,0.53}{#1}}
\newcommand{\StringTok}[1]{\textcolor[rgb]{0.25,0.44,0.63}{#1}}
\newcommand{\VariableTok}[1]{\textcolor[rgb]{0.10,0.09,0.49}{#1}}
\newcommand{\VerbatimStringTok}[1]{\textcolor[rgb]{0.25,0.44,0.63}{#1}}
\newcommand{\WarningTok}[1]{\textcolor[rgb]{0.38,0.63,0.69}{\textbf{\textit{#1}}}}
\usepackage{graphicx}
\makeatletter
\def\maxwidth{\ifdim\Gin@nat@width>\linewidth\linewidth\else\Gin@nat@width\fi}
\def\maxheight{\ifdim\Gin@nat@height>\textheight\textheight\else\Gin@nat@height\fi}
\makeatother
% Scale images if necessary, so that they will not overflow the page
% margins by default, and it is still possible to overwrite the defaults
% using explicit options in \includegraphics[width, height, ...]{}
\setkeys{Gin}{width=\maxwidth,height=\maxheight,keepaspectratio}
% Set default figure placement to htbp
\makeatletter
\def\fps@figure{htbp}
\makeatother
\setlength{\emergencystretch}{3em} % prevent overfull lines
\providecommand{\tightlist}{%
  \setlength{\itemsep}{0pt}\setlength{\parskip}{0pt}}
\setcounter{secnumdepth}{5}
\RecustomVerbatimEnvironment{Highlighting}{Verbatim}{commandchars=\\\{\},fontfamily=mlmr,frame=leftline,numbers=left,numbersep=2.5pt}

\setlength{\fboxsep}{5pt}
\setlength{\columnsep}{20pt}

\setkomafont{title}{\normalfont\sffamily}
\setkomafont{disposition}{\normalfont\sffamily}
\setkomafont{subtitle}{\normalfont\large\sffamily}
\setkomafont{section}{\normalfont\Large\sffamily}
\setkomafont{subsection}{\normalfont\large\sffamily}

\titlehead{Universidad Simón Bolívar\hfill Matemáticas Puras y Aplicadas}
\usepackage{mlmodern}
\usepackage{plex-mono}
\ifxetex
  % Load polyglossia as late as possible: uses bidi with RTL langages (e.g. Hebrew, Arabic)
  \usepackage{polyglossia}
  \setmainlanguage[]{spanish}
\else
  \usepackage[shorthands=off,main=spanish]{babel}
\fi

\title{Tarea 2}
\usepackage{etoolbox}
\makeatletter
\providecommand{\subtitle}[1]{% add subtitle to \maketitle
  \apptocmd{\@title}{\par {\large #1 \par}}{}{}
}
\makeatother
\subtitle{Topología y Geometría II}
\author{Jhonny Lanzuisi}
\date{9 de Junio de 2022}

\begin{document}
\maketitle

{
\setcounter{tocdepth}{3}
\tableofcontents
}
\begin{Shaded}
\begin{Highlighting}[]
\CommentTok{\# Configuración de python y gráficos}
\ImportTok{import}\NormalTok{ sympy }\ImportTok{as}\NormalTok{ sym}
\ImportTok{import}\NormalTok{ numpy }\ImportTok{as}\NormalTok{ np}
\ImportTok{import}\NormalTok{ matplotlib.style}
\ImportTok{import}\NormalTok{ matplotlib.pyplot }\ImportTok{as}\NormalTok{ plt}
\ImportTok{import}\NormalTok{ matplotlib}

\NormalTok{matplotlib.style.use(}\StringTok{\textquotesingle{}seaborn\textquotesingle{}}\NormalTok{)}

\NormalTok{matplotlib.rcParams.update(\{}
        \StringTok{\textquotesingle{}figure.autolayout\textquotesingle{}}\NormalTok{: }\VariableTok{True}\NormalTok{,}
\NormalTok{\})}

\KeywordTok{def}\NormalTok{ lp(input\_list):}
    \ControlFlowTok{for}\NormalTok{ i }\KeywordTok{in}\NormalTok{ input\_list:}
        \BuiltInTok{print}\NormalTok{(}\StringTok{\textquotesingle{}$$\textquotesingle{}} \OperatorTok{+}\NormalTok{ sym.latex(i) }\OperatorTok{+} \StringTok{\textquotesingle{}$$\textquotesingle{}}\NormalTok{)}
\end{Highlighting}
\end{Shaded}

\hypertarget{primera-pregunta}{%
\section{Primera pregunta}\label{primera-pregunta}}

Sea \(f(x, y, z) = xyz^2\).

\hypertarget{puntos-y-valores-cruxedticos}{%
\subsection{Puntos y valores
críticos}\label{puntos-y-valores-cruxedticos}}

Calculamos primero el diferencial, los puntos donde este no sea
sobreyectivo son son puntos críticos. Las derivadas parciales de \(f\)
son:

\begin{Shaded}
\begin{Highlighting}[]
\NormalTok{x,y,z,f }\OperatorTok{=}\NormalTok{ sym.symbols(}\StringTok{\textquotesingle{}x,y,z,f\textquotesingle{}}\NormalTok{)}

\NormalTok{f }\OperatorTok{=}\NormalTok{ x}\OperatorTok{*}\NormalTok{y}\OperatorTok{*}\NormalTok{z}\OperatorTok{**}\DecValTok{2}

\NormalTok{derivatives }\OperatorTok{=}\NormalTok{ [}
\NormalTok{    sym.Derivative(f,x),}
\NormalTok{    sym.Derivative(f,y),}
\NormalTok{    sym.Derivative(f,z)}
\NormalTok{]}

\NormalTok{lp(}
\NormalTok{    [}
\NormalTok{        sym.Eq(derivatives[}\DecValTok{0}\NormalTok{], derivatives[}\DecValTok{0}\NormalTok{].doit()),}
\NormalTok{        sym.Eq(derivatives[}\DecValTok{1}\NormalTok{], derivatives[}\DecValTok{1}\NormalTok{].doit()),}
\NormalTok{        sym.Eq(derivatives[}\DecValTok{2}\NormalTok{], derivatives[}\DecValTok{2}\NormalTok{].doit())}
\NormalTok{    ]}
\NormalTok{)}
\end{Highlighting}
\end{Shaded}

\[\frac{\partial}{\partial x} x y z^{2} = y z^{2}\]
\[\frac{\partial}{\partial y} x y z^{2} = x z^{2}\]
\[\frac{\partial}{\partial z} x y z^{2} = 2 x y z\]

El diferencial esta dado, en la base canonica, por:

\begin{Shaded}
\begin{Highlighting}[]
\NormalTok{df }\OperatorTok{=}\NormalTok{ sym.Matrix([}
\NormalTok{    derivatives[}\DecValTok{0}\NormalTok{],}
\NormalTok{    derivatives[}\DecValTok{1}\NormalTok{],}
\NormalTok{    derivatives[}\DecValTok{2}\NormalTok{]}
\NormalTok{])}
\NormalTok{lp([df.doit()])}
\end{Highlighting}
\end{Shaded}

\[\left[\begin{matrix}y z^{2}\\x z^{2}\\2 x y z\end{matrix}\right]\]

Entonces los puntos críticos son los \((x,y,z)\in \mathbb{R}^3\) de la
forma: \[
    \begin{pmatrix}
    x, y, 0
    \end{pmatrix} \;\text{o} 
    \begin{pmatrix}
    0, 0, z
    \end{pmatrix} \;\text{o}
    \begin{pmatrix}
    0, 0, 0
    \end{pmatrix}.
\] Y los valores críticos son la imagen de \(f\) sobre las 3 clases de
puntos anteriores, en este caso solamente el cero puesto que
\(f(x, y, 0) = f(0, 0, z) = f(0, 0, 0) = 0\).

\hypertarget{superficie-regular}{%
\subsection{Superficie regular}\label{superficie-regular}}

La superficie es regular para todo punto \(c\) que no sea un valor
crítico, por lo que \(f(x,y,z)=c\) es regular siempre que \(c\neq0\).

\hypertarget{segunda-pregunta}{%
\section{Segunda pregunta}\label{segunda-pregunta}}

Sea \(X(u,v)= (u+v, u-v, 4uv)\).

\hypertarget{s-es-superficie-regular}{%
\subsection{\texorpdfstring{\(S\) es superficie
regular}{S es superficie regular}}\label{s-es-superficie-regular}}

Consideremos la funcion \(f\colon \mathbb{R}^3\to \mathbb{R}\) dada por
\[
  f(x,y,z) = x^2 - y^2 - z.
\] Entonces \(S\) será una superficie regular siempre que \(0\) no sea
un valor crítico de \(f\).

Para verficar esto último, calculemos las derivadas parciales:

\begin{Shaded}
\begin{Highlighting}[]
\NormalTok{f }\OperatorTok{=}\NormalTok{ x}\OperatorTok{**}\DecValTok{2} \OperatorTok{{-}}\NormalTok{ y}\OperatorTok{**}\DecValTok{2} \OperatorTok{{-}}\NormalTok{ z}

\NormalTok{derivatives }\OperatorTok{=}\NormalTok{ [}
\NormalTok{    sym.Derivative(f,x),}
\NormalTok{    sym.Derivative(f,y),}
\NormalTok{    sym.Derivative(f,z)}
\NormalTok{]}

\NormalTok{lp([}
\NormalTok{    sym.Eq(derivatives[}\DecValTok{0}\NormalTok{], derivatives[}\DecValTok{0}\NormalTok{].doit()),}
\NormalTok{    sym.Eq(derivatives[}\DecValTok{1}\NormalTok{], derivatives[}\DecValTok{1}\NormalTok{].doit()),}
\NormalTok{    sym.Eq(derivatives[}\DecValTok{2}\NormalTok{], derivatives[}\DecValTok{2}\NormalTok{].doit())}
\NormalTok{])}
\end{Highlighting}
\end{Shaded}

\[\frac{\partial}{\partial x} \left(x^{2} - y^{2} - z\right) = 2 x\]
\[\frac{\partial}{\partial y} \left(x^{2} - y^{2} - z\right) = - 2 y\]
\[\frac{\partial}{\partial z} \left(x^{2} - y^{2} - z\right) = -1\]

Entonces \(f\) no tiene puntos ni valores críticos, puesto que el
diferencial nunca se hace cero. En particular \(0\) no es un valor
crítico y \(S\) es regular.

\hypertarget{x-es-parametrizaciuxf3n-de-s}{%
\subsection{\texorpdfstring{\(X\) es parametrización de
\(S\)}{X es parametrización de S}}\label{x-es-parametrizaciuxf3n-de-s}}

Veamos primero el grafico de \(X\) en la figura \ref{p1}.

\begin{Shaded}
\begin{Highlighting}[]
\NormalTok{u,v }\OperatorTok{=}\NormalTok{ sym.symbols(}\StringTok{\textquotesingle{}u,v\textquotesingle{}}\NormalTok{)}

\NormalTok{x }\OperatorTok{=}\NormalTok{ u}\OperatorTok{+}\NormalTok{v}
\NormalTok{y }\OperatorTok{=}\NormalTok{ u}\OperatorTok{{-}}\NormalTok{v}
\NormalTok{z }\OperatorTok{=} \DecValTok{4}\OperatorTok{*}\NormalTok{u}\OperatorTok{*}\NormalTok{v}

\NormalTok{p1 }\OperatorTok{=}\NormalTok{ sym.plotting.plot3d\_parametric\_surface(}
\NormalTok{    x,y,z, show}\OperatorTok{=}\VariableTok{False}
\NormalTok{)}
\NormalTok{p1.save(}\StringTok{\textquotesingle{}p1.pdf\textquotesingle{}}\NormalTok{)}
\end{Highlighting}
\end{Shaded}

Para verificar que \(X\) es una parametrizacion hay que ver que cumpla
con las 3 condiciones de la definicion.

\hypertarget{x-es-diferenciable}{%
\subsubsection{\texorpdfstring{\(X\) es
diferenciable}{X es diferenciable}}\label{x-es-diferenciable}}

Consideremos las funciones componentes de \(X\): \[
x(u,v) = u+v,\quad y(u,v) = u-v,\quad z(u,v) = 4uv.
\] Estas funciones son \(C^\infty\) puesto que son polinomicas. Se sigue
entonces que \(X\) es diferenciable.

\hypertarget{x-es-homeomorfismo}{%
\subsubsection{\texorpdfstring{\(X\) es
homeomorfismo}{X es homeomorfismo}}\label{x-es-homeomorfismo}}

Basta con ver que \(X^{-1}\colon \mathbb{R}^3\to\mathbb{R}^2\) es
continua.

Primeros, hallemos la función inversa. Esto equivale a resolver el
siguiente sistema de ecuaciones: \[
    x = u+v,\quad y = u-v,\quad z = 4uv,\quad z=x^2-y^2.
\]

De las primeras dos ecuaciones tenemos que \[
    u = \frac{x+y}{2},
\] y sustituyendo en la expresión para \(z\) se obtiene: \[
    v = \frac{z}{2(x+y)},
\] finalmente, al sustituir \(z=x^2-y^2\) en la ecuación anterior \[
    v = \frac{x-y}{2}
\]

Entonces la función inversa \(X^{-1}\) viene dada por: \[
    X^{-1}(x,y,z) = \left( \frac{x+y}{2} , \frac{x-y}{2} \right).
\]

Esta función es continua puesto que sus componentes son funciones
continuas. Luego, \(X\) es homeomorfismo.

\hypertarget{x-es-regular}{%
\subsubsection{\texorpdfstring{\(X\) es
regular}{X es regular}}\label{x-es-regular}}

La condición de regularidad equivale a que las columnas del diferencial
de \(X\) sean linealmente independientes. Calculemos el diferencial.

\begin{Shaded}
\begin{Highlighting}[]
\NormalTok{u,v }\OperatorTok{=}\NormalTok{ sym.symbols(}\StringTok{\textquotesingle{}u,v\textquotesingle{}}\NormalTok{)}

\NormalTok{dX }\OperatorTok{=}\NormalTok{ sym.Matrix([}
\NormalTok{    [x.diff(u), x.diff(v)],}
\NormalTok{    [y.diff(u), y.diff(v)],}
\NormalTok{    [z.diff(u), z.diff(v)],}
\NormalTok{])}

\NormalTok{lp([dX])}
\end{Highlighting}
\end{Shaded}

\[\left[\begin{matrix}1 & 1\\1 & -1\\4 v & 4 u\end{matrix}\right]\]

Como dos vectores son linealmente independientes si uno no es múltiplo
escalar del otro, se sigue que las columnas del diferencial son
independientes, pues el \(-1\) en la segunda columna hace que siempre
difieran por un signo, y que \(X\) es regular.

Se tiene entonces que \(X\) es una parametrización de \(S\).

\onecolumn
\appendix

\hypertarget{gruxe1ficos}{%
\section{Gráficos}\label{gruxe1ficos}}

\begin{figure}
\includegraphics[width=1\linewidth]{p1} \caption{\label{p1} Gráfico de $X(u,v)=(u+v, u-v, 4uv)$}\label{fig:unnamed-chunk-7}
\end{figure}

\end{document}
