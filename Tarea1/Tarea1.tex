\documentclass[12pt]{scrartcl}

    \usepackage[breakable]{tcolorbox}
    

    % Basic figure setup, for now with no caption control since it's done
    % automatically by Pandoc (which extracts ![](path) syntax from Markdown).
    \usepackage{graphicx}
    % Maintain compatibility with old templates. Remove in nbconvert 6.0
    \let\Oldincludegraphics\includegraphics
    % Ensure that by default, figures have no caption (until we provide a
    % proper Figure object with a Caption API and a way to capture that
    % in the conversion process - todo).
    \usepackage{caption}
    \DeclareCaptionFormat{nocaption}{}
    \captionsetup{format=nocaption,aboveskip=0pt,belowskip=0pt}

    \usepackage{float}
    \floatplacement{figure}{H} % forces figures to be placed at the correct location
    \usepackage{xcolor} % Allow colors to be defined
    \usepackage{enumerate} % Needed for markdown enumerations to work
    \usepackage{geometry} % Used to adjust the document margins
    \usepackage{amsmath} % Equations
    \usepackage{amssymb} % Equations
    \usepackage{textcomp} % defines textquotesingle
    % Hack from http://tex.stackexchange.com/a/47451/13684:
    \AtBeginDocument{%
        \def\PYZsq{\textquotesingle}% Upright quotes in Pygmentized code
    }
    \usepackage{upquote} % Upright quotes for verbatim code
    \usepackage{eurosym} % defines \euro

    \usepackage{iftex}
    \ifPDFTeX
        \usepackage[T1]{fontenc}
        \IfFileExists{alphabeta.sty}{
              \usepackage{alphabeta}
          }{
              \usepackage[mathletters]{ucs}
              \usepackage[utf8x]{inputenc}
          }
    \else
        \usepackage{fontspec}
        \usepackage{unicode-math}
    \fi
\usepackage{mlmodern}

    \usepackage{fancyvrb} % verbatim replacement that allows latex
    \usepackage{grffile} % extends the file name processing of package graphics
                         % to support a larger range
    \makeatletter % fix for old versions of grffile with XeLaTeX
    \@ifpackagelater{grffile}{2019/11/01}
    {
      % Do nothing on new versions
    }
    {
      \def\Gread@@xetex#1{%
        \IfFileExists{"\Gin@base".bb}%
        {\Gread@eps{\Gin@base.bb}}%
        {\Gread@@xetex@aux#1}%
      }
    }
    \makeatother
    \usepackage[Export]{adjustbox} % Used to constrain images to a maximum size
    \adjustboxset{max size={0.9\linewidth}{0.9\paperheight}}

    % The hyperref package gives us a pdf with properly built
    % internal navigation ('pdf bookmarks' for the table of contents,
    % internal cross-reference links, web links for URLs, etc.)
    \usepackage{hyperref}
    % The default LaTeX title has an obnoxious amount of whitespace. By default,
    \usepackage{longtable} % longtable support required by pandoc >1.10
    \usepackage{booktabs}  % table support for pandoc > 1.12.2
    \usepackage{array}     % table support for pandoc >= 2.11.3
    \usepackage{calc}      % table minipage width calculation for pandoc >= 2.11.1
    \usepackage[inline]{enumitem} % IRkernel/repr support (it uses the enumerate* environment)
    \usepackage[normalem]{ulem} % ulem is needed to support strikethroughs (\sout)
                                % normalem makes italics be italics, not underlines
    \usepackage{mathrsfs}
    

    
    % Colors for the hyperref package
    \definecolor{urlcolor}{rgb}{0,.145,.698}
    \definecolor{linkcolor}{rgb}{.71,0.21,0.01}
    \definecolor{citecolor}{rgb}{.12,.54,.11}

    % ANSI colors
    \definecolor{ansi-black}{HTML}{3E424D}
    \definecolor{ansi-black-intense}{HTML}{282C36}
    \definecolor{ansi-red}{HTML}{E75C58}
    \definecolor{ansi-red-intense}{HTML}{B22B31}
    \definecolor{ansi-green}{HTML}{00A250}
    \definecolor{ansi-green-intense}{HTML}{007427}
    \definecolor{ansi-yellow}{HTML}{DDB62B}
    \definecolor{ansi-yellow-intense}{HTML}{B27D12}
    \definecolor{ansi-blue}{HTML}{208FFB}
    \definecolor{ansi-blue-intense}{HTML}{0065CA}
    \definecolor{ansi-magenta}{HTML}{D160C4}
    \definecolor{ansi-magenta-intense}{HTML}{A03196}
    \definecolor{ansi-cyan}{HTML}{60C6C8}
    \definecolor{ansi-cyan-intense}{HTML}{258F8F}
    \definecolor{ansi-white}{HTML}{C5C1B4}
    \definecolor{ansi-white-intense}{HTML}{A1A6B2}
    \definecolor{ansi-default-inverse-fg}{HTML}{FFFFFF}
    \definecolor{ansi-default-inverse-bg}{HTML}{000000}

    % common color for the border for error outputs.
    \definecolor{outerrorbackground}{HTML}{FFDFDF}

    % commands and environments needed by pandoc snippets
    % extracted from the output of `pandoc -s`
    \providecommand{\tightlist}{%
      \setlength{\itemsep}{0pt}\setlength{\parskip}{0pt}}
    \DefineVerbatimEnvironment{Highlighting}{Verbatim}{commandchars=\\\{\}}
    % Add ',fontsize=\small' for more characters per line
    \newenvironment{Shaded}{}{}
    \newcommand{\KeywordTok}[1]{\textcolor[rgb]{0.00,0.44,0.13}{\textbf{{#1}}}}
    \newcommand{\DataTypeTok}[1]{\textcolor[rgb]{0.56,0.13,0.00}{{#1}}}
    \newcommand{\DecValTok}[1]{\textcolor[rgb]{0.25,0.63,0.44}{{#1}}}
    \newcommand{\BaseNTok}[1]{\textcolor[rgb]{0.25,0.63,0.44}{{#1}}}
    \newcommand{\FloatTok}[1]{\textcolor[rgb]{0.25,0.63,0.44}{{#1}}}
    \newcommand{\CharTok}[1]{\textcolor[rgb]{0.25,0.44,0.63}{{#1}}}
    \newcommand{\StringTok}[1]{\textcolor[rgb]{0.25,0.44,0.63}{{#1}}}
    \newcommand{\CommentTok}[1]{\textcolor[rgb]{0.38,0.63,0.69}{\textit{{#1}}}}
    \newcommand{\OtherTok}[1]{\textcolor[rgb]{0.00,0.44,0.13}{{#1}}}
    \newcommand{\AlertTok}[1]{\textcolor[rgb]{1.00,0.00,0.00}{\textbf{{#1}}}}
    \newcommand{\FunctionTok}[1]{\textcolor[rgb]{0.02,0.16,0.49}{{#1}}}
    \newcommand{\RegionMarkerTok}[1]{{#1}}
    \newcommand{\ErrorTok}[1]{\textcolor[rgb]{1.00,0.00,0.00}{\textbf{{#1}}}}
    \newcommand{\NormalTok}[1]{{#1}}

    % Additional commands for more recent versions of Pandoc
    \newcommand{\ConstantTok}[1]{\textcolor[rgb]{0.53,0.00,0.00}{{#1}}}
    \newcommand{\SpecialCharTok}[1]{\textcolor[rgb]{0.25,0.44,0.63}{{#1}}}
    \newcommand{\VerbatimStringTok}[1]{\textcolor[rgb]{0.25,0.44,0.63}{{#1}}}
    \newcommand{\SpecialStringTok}[1]{\textcolor[rgb]{0.73,0.40,0.53}{{#1}}}
    \newcommand{\ImportTok}[1]{{#1}}
    \newcommand{\DocumentationTok}[1]{\textcolor[rgb]{0.73,0.13,0.13}{\textit{{#1}}}}
    \newcommand{\AnnotationTok}[1]{\textcolor[rgb]{0.38,0.63,0.69}{\textbf{\textit{{#1}}}}}
    \newcommand{\CommentVarTok}[1]{\textcolor[rgb]{0.38,0.63,0.69}{\textbf{\textit{{#1}}}}}
    \newcommand{\VariableTok}[1]{\textcolor[rgb]{0.10,0.09,0.49}{{#1}}}
    \newcommand{\ControlFlowTok}[1]{\textcolor[rgb]{0.00,0.44,0.13}{\textbf{{#1}}}}
    \newcommand{\OperatorTok}[1]{\textcolor[rgb]{0.40,0.40,0.40}{{#1}}}
    \newcommand{\BuiltInTok}[1]{{#1}}
    \newcommand{\ExtensionTok}[1]{{#1}}
    \newcommand{\PreprocessorTok}[1]{\textcolor[rgb]{0.74,0.48,0.00}{{#1}}}
    \newcommand{\AttributeTok}[1]{\textcolor[rgb]{0.49,0.56,0.16}{{#1}}}
    \newcommand{\InformationTok}[1]{\textcolor[rgb]{0.38,0.63,0.69}{\textbf{\textit{{#1}}}}}
    \newcommand{\WarningTok}[1]{\textcolor[rgb]{0.38,0.63,0.69}{\textbf{\textit{{#1}}}}}


    % Define a nice break command that doesn't care if a line doesn't already
    % exist.
    \def\br{\hspace*{\fill} \\* }
    % Math Jax compatibility definitions
    \def\gt{>}
    \def\lt{<}
    \let\Oldtex\TeX
    \let\Oldlatex\LaTeX
    \renewcommand{\TeX}{\textrm{\Oldtex}}
    \renewcommand{\LaTeX}{\textrm{\Oldlatex}}
    % Document parameters
    % Document title
    \newcommand{\MyTitle}{Primera Tarea}
\titlehead{Matemáticas puras y aplicadas\hfill Universidad Simón Bolívar}
\title{\MyTitle}
\author{Jhonny Lanzusi}
\date{\today}
\subject{Topología y Geometría II}
\subtitle{Curvas paramétricas, curvatura, torsión}

\KOMAoptions{parskip=half}

\usepackage[tracking = true]{microtype}
\SetTracking{encoding = *, shape = sc}{40}

\setkomafont{title}{\normalfont}
\setkomafont{subtitle}{\normalfont\itshape}
\setkomafont{subject}{\normalfont\scshape}
\setkomafont{section}{\normalfont\scshape}
\setkomafont{subsection}{\normalfont}

\usepackage{scrlayer-scrpage}
\pagestyle{scrheadings}

\chead{\MyTitle}

    
    
    
    
    
% Pygments definitions
\makeatletter
\def\PY@reset{\let\PY@it=\relax \let\PY@bf=\relax%
    \let\PY@ul=\relax \let\PY@tc=\relax%
    \let\PY@bc=\relax \let\PY@ff=\relax}
\def\PY@tok#1{\csname PY@tok@#1\endcsname}
\def\PY@toks#1+{\ifx\relax#1\empty\else%
    \PY@tok{#1}\expandafter\PY@toks\fi}
\def\PY@do#1{\PY@bc{\PY@tc{\PY@ul{%
    \PY@it{\PY@bf{\PY@ff{#1}}}}}}}
\def\PY#1#2{\PY@reset\PY@toks#1+\relax+\PY@do{#2}}

\@namedef{PY@tok@w}{\def\PY@tc##1{\textcolor[rgb]{0.73,0.73,0.73}{##1}}}
\@namedef{PY@tok@c}{\let\PY@it=\textit\def\PY@tc##1{\textcolor[rgb]{0.24,0.48,0.48}{##1}}}
\@namedef{PY@tok@cp}{\def\PY@tc##1{\textcolor[rgb]{0.61,0.40,0.00}{##1}}}
\@namedef{PY@tok@k}{\let\PY@bf=\textbf\def\PY@tc##1{\textcolor[rgb]{0.00,0.50,0.00}{##1}}}
\@namedef{PY@tok@kp}{\def\PY@tc##1{\textcolor[rgb]{0.00,0.50,0.00}{##1}}}
\@namedef{PY@tok@kt}{\def\PY@tc##1{\textcolor[rgb]{0.69,0.00,0.25}{##1}}}
\@namedef{PY@tok@o}{\def\PY@tc##1{\textcolor[rgb]{0.40,0.40,0.40}{##1}}}
\@namedef{PY@tok@ow}{\let\PY@bf=\textbf\def\PY@tc##1{\textcolor[rgb]{0.67,0.13,1.00}{##1}}}
\@namedef{PY@tok@nb}{\def\PY@tc##1{\textcolor[rgb]{0.00,0.50,0.00}{##1}}}
\@namedef{PY@tok@nf}{\def\PY@tc##1{\textcolor[rgb]{0.00,0.00,1.00}{##1}}}
\@namedef{PY@tok@nc}{\let\PY@bf=\textbf\def\PY@tc##1{\textcolor[rgb]{0.00,0.00,1.00}{##1}}}
\@namedef{PY@tok@nn}{\let\PY@bf=\textbf\def\PY@tc##1{\textcolor[rgb]{0.00,0.00,1.00}{##1}}}
\@namedef{PY@tok@ne}{\let\PY@bf=\textbf\def\PY@tc##1{\textcolor[rgb]{0.80,0.25,0.22}{##1}}}
\@namedef{PY@tok@nv}{\def\PY@tc##1{\textcolor[rgb]{0.10,0.09,0.49}{##1}}}
\@namedef{PY@tok@no}{\def\PY@tc##1{\textcolor[rgb]{0.53,0.00,0.00}{##1}}}
\@namedef{PY@tok@nl}{\def\PY@tc##1{\textcolor[rgb]{0.46,0.46,0.00}{##1}}}
\@namedef{PY@tok@ni}{\let\PY@bf=\textbf\def\PY@tc##1{\textcolor[rgb]{0.44,0.44,0.44}{##1}}}
\@namedef{PY@tok@na}{\def\PY@tc##1{\textcolor[rgb]{0.41,0.47,0.13}{##1}}}
\@namedef{PY@tok@nt}{\let\PY@bf=\textbf\def\PY@tc##1{\textcolor[rgb]{0.00,0.50,0.00}{##1}}}
\@namedef{PY@tok@nd}{\def\PY@tc##1{\textcolor[rgb]{0.67,0.13,1.00}{##1}}}
\@namedef{PY@tok@s}{\def\PY@tc##1{\textcolor[rgb]{0.73,0.13,0.13}{##1}}}
\@namedef{PY@tok@sd}{\let\PY@it=\textit\def\PY@tc##1{\textcolor[rgb]{0.73,0.13,0.13}{##1}}}
\@namedef{PY@tok@si}{\let\PY@bf=\textbf\def\PY@tc##1{\textcolor[rgb]{0.64,0.35,0.47}{##1}}}
\@namedef{PY@tok@se}{\let\PY@bf=\textbf\def\PY@tc##1{\textcolor[rgb]{0.67,0.36,0.12}{##1}}}
\@namedef{PY@tok@sr}{\def\PY@tc##1{\textcolor[rgb]{0.64,0.35,0.47}{##1}}}
\@namedef{PY@tok@ss}{\def\PY@tc##1{\textcolor[rgb]{0.10,0.09,0.49}{##1}}}
\@namedef{PY@tok@sx}{\def\PY@tc##1{\textcolor[rgb]{0.00,0.50,0.00}{##1}}}
\@namedef{PY@tok@m}{\def\PY@tc##1{\textcolor[rgb]{0.40,0.40,0.40}{##1}}}
\@namedef{PY@tok@gh}{\let\PY@bf=\textbf\def\PY@tc##1{\textcolor[rgb]{0.00,0.00,0.50}{##1}}}
\@namedef{PY@tok@gu}{\let\PY@bf=\textbf\def\PY@tc##1{\textcolor[rgb]{0.50,0.00,0.50}{##1}}}
\@namedef{PY@tok@gd}{\def\PY@tc##1{\textcolor[rgb]{0.63,0.00,0.00}{##1}}}
\@namedef{PY@tok@gi}{\def\PY@tc##1{\textcolor[rgb]{0.00,0.52,0.00}{##1}}}
\@namedef{PY@tok@gr}{\def\PY@tc##1{\textcolor[rgb]{0.89,0.00,0.00}{##1}}}
\@namedef{PY@tok@ge}{\let\PY@it=\textit}
\@namedef{PY@tok@gs}{\let\PY@bf=\textbf}
\@namedef{PY@tok@gp}{\let\PY@bf=\textbf\def\PY@tc##1{\textcolor[rgb]{0.00,0.00,0.50}{##1}}}
\@namedef{PY@tok@go}{\def\PY@tc##1{\textcolor[rgb]{0.44,0.44,0.44}{##1}}}
\@namedef{PY@tok@gt}{\def\PY@tc##1{\textcolor[rgb]{0.00,0.27,0.87}{##1}}}
\@namedef{PY@tok@err}{\def\PY@bc##1{{\setlength{\fboxsep}{\string -\fboxrule}\fcolorbox[rgb]{1.00,0.00,0.00}{1,1,1}{\strut ##1}}}}
\@namedef{PY@tok@kc}{\let\PY@bf=\textbf\def\PY@tc##1{\textcolor[rgb]{0.00,0.50,0.00}{##1}}}
\@namedef{PY@tok@kd}{\let\PY@bf=\textbf\def\PY@tc##1{\textcolor[rgb]{0.00,0.50,0.00}{##1}}}
\@namedef{PY@tok@kn}{\let\PY@bf=\textbf\def\PY@tc##1{\textcolor[rgb]{0.00,0.50,0.00}{##1}}}
\@namedef{PY@tok@kr}{\let\PY@bf=\textbf\def\PY@tc##1{\textcolor[rgb]{0.00,0.50,0.00}{##1}}}
\@namedef{PY@tok@bp}{\def\PY@tc##1{\textcolor[rgb]{0.00,0.50,0.00}{##1}}}
\@namedef{PY@tok@fm}{\def\PY@tc##1{\textcolor[rgb]{0.00,0.00,1.00}{##1}}}
\@namedef{PY@tok@vc}{\def\PY@tc##1{\textcolor[rgb]{0.10,0.09,0.49}{##1}}}
\@namedef{PY@tok@vg}{\def\PY@tc##1{\textcolor[rgb]{0.10,0.09,0.49}{##1}}}
\@namedef{PY@tok@vi}{\def\PY@tc##1{\textcolor[rgb]{0.10,0.09,0.49}{##1}}}
\@namedef{PY@tok@vm}{\def\PY@tc##1{\textcolor[rgb]{0.10,0.09,0.49}{##1}}}
\@namedef{PY@tok@sa}{\def\PY@tc##1{\textcolor[rgb]{0.73,0.13,0.13}{##1}}}
\@namedef{PY@tok@sb}{\def\PY@tc##1{\textcolor[rgb]{0.73,0.13,0.13}{##1}}}
\@namedef{PY@tok@sc}{\def\PY@tc##1{\textcolor[rgb]{0.73,0.13,0.13}{##1}}}
\@namedef{PY@tok@dl}{\def\PY@tc##1{\textcolor[rgb]{0.73,0.13,0.13}{##1}}}
\@namedef{PY@tok@s2}{\def\PY@tc##1{\textcolor[rgb]{0.73,0.13,0.13}{##1}}}
\@namedef{PY@tok@sh}{\def\PY@tc##1{\textcolor[rgb]{0.73,0.13,0.13}{##1}}}
\@namedef{PY@tok@s1}{\def\PY@tc##1{\textcolor[rgb]{0.73,0.13,0.13}{##1}}}
\@namedef{PY@tok@mb}{\def\PY@tc##1{\textcolor[rgb]{0.40,0.40,0.40}{##1}}}
\@namedef{PY@tok@mf}{\def\PY@tc##1{\textcolor[rgb]{0.40,0.40,0.40}{##1}}}
\@namedef{PY@tok@mh}{\def\PY@tc##1{\textcolor[rgb]{0.40,0.40,0.40}{##1}}}
\@namedef{PY@tok@mi}{\def\PY@tc##1{\textcolor[rgb]{0.40,0.40,0.40}{##1}}}
\@namedef{PY@tok@il}{\def\PY@tc##1{\textcolor[rgb]{0.40,0.40,0.40}{##1}}}
\@namedef{PY@tok@mo}{\def\PY@tc##1{\textcolor[rgb]{0.40,0.40,0.40}{##1}}}
\@namedef{PY@tok@ch}{\let\PY@it=\textit\def\PY@tc##1{\textcolor[rgb]{0.24,0.48,0.48}{##1}}}
\@namedef{PY@tok@cm}{\let\PY@it=\textit\def\PY@tc##1{\textcolor[rgb]{0.24,0.48,0.48}{##1}}}
\@namedef{PY@tok@cpf}{\let\PY@it=\textit\def\PY@tc##1{\textcolor[rgb]{0.24,0.48,0.48}{##1}}}
\@namedef{PY@tok@c1}{\let\PY@it=\textit\def\PY@tc##1{\textcolor[rgb]{0.24,0.48,0.48}{##1}}}
\@namedef{PY@tok@cs}{\let\PY@it=\textit\def\PY@tc##1{\textcolor[rgb]{0.24,0.48,0.48}{##1}}}

\def\PYZbs{\char`\\}
\def\PYZus{\char`\_}
\def\PYZob{\char`\{}
\def\PYZcb{\char`\}}
\def\PYZca{\char`\^}
\def\PYZam{\char`\&}
\def\PYZlt{\char`\<}
\def\PYZgt{\char`\>}
\def\PYZsh{\char`\#}
\def\PYZpc{\char`\%}
\def\PYZdl{\char`\$}
\def\PYZhy{\char`\-}
\def\PYZsq{\char`\'}
\def\PYZdq{\char`\"}
\def\PYZti{\char`\~}
% for compatibility with earlier versions
\def\PYZat{@}
\def\PYZlb{[}
\def\PYZrb{]}
\makeatother


    % For linebreaks inside Verbatim environment from package fancyvrb.
    \makeatletter
        \newbox\Wrappedcontinuationbox
        \newbox\Wrappedvisiblespacebox
        \newcommand*\Wrappedvisiblespace {\textcolor{red}{\textvisiblespace}}
        \newcommand*\Wrappedcontinuationsymbol {\textcolor{red}{\llap{\tiny$\m@th\hookrightarrow$}}}
        \newcommand*\Wrappedcontinuationindent {3ex }
        \newcommand*\Wrappedafterbreak {\kern\Wrappedcontinuationindent\copy\Wrappedcontinuationbox}
        % Take advantage of the already applied Pygments mark-up to insert
        % potential linebreaks for TeX processing.
        %        {, <, #, %, $, ' and ": go to next line.
        %        _, }, ^, &, >, - and ~: stay at end of broken line.
        % Use of \textquotesingle for straight quote.
        \newcommand*\Wrappedbreaksatspecials {%
            \def\PYGZus{\discretionary{\char`\_}{\Wrappedafterbreak}{\char`\_}}%
            \def\PYGZob{\discretionary{}{\Wrappedafterbreak\char`\{}{\char`\{}}%
            \def\PYGZcb{\discretionary{\char`\}}{\Wrappedafterbreak}{\char`\}}}%
            \def\PYGZca{\discretionary{\char`\^}{\Wrappedafterbreak}{\char`\^}}%
            \def\PYGZam{\discretionary{\char`\&}{\Wrappedafterbreak}{\char`\&}}%
            \def\PYGZlt{\discretionary{}{\Wrappedafterbreak\char`\<}{\char`\<}}%
            \def\PYGZgt{\discretionary{\char`\>}{\Wrappedafterbreak}{\char`\>}}%
            \def\PYGZsh{\discretionary{}{\Wrappedafterbreak\char`\#}{\char`\#}}%
            \def\PYGZpc{\discretionary{}{\Wrappedafterbreak\char`\%}{\char`\%}}%
            \def\PYGZdl{\discretionary{}{\Wrappedafterbreak\char`\$}{\char`\$}}%
            \def\PYGZhy{\discretionary{\char`\-}{\Wrappedafterbreak}{\char`\-}}%
            \def\PYGZsq{\discretionary{}{\Wrappedafterbreak\textquotesingle}{\textquotesingle}}%
            \def\PYGZdq{\discretionary{}{\Wrappedafterbreak\char`\"}{\char`\"}}%
            \def\PYGZti{\discretionary{\char`\~}{\Wrappedafterbreak}{\char`\~}}%
        }
        % Some characters . , ; ? ! / are not pygmentized.
        % This macro makes them "active" and they will insert potential linebreaks
        \newcommand*\Wrappedbreaksatpunct {%
            \lccode`\~`\.\lowercase{\def~}{\discretionary{\hbox{\char`\.}}{\Wrappedafterbreak}{\hbox{\char`\.}}}%
            \lccode`\~`\,\lowercase{\def~}{\discretionary{\hbox{\char`\,}}{\Wrappedafterbreak}{\hbox{\char`\,}}}%
            \lccode`\~`\;\lowercase{\def~}{\discretionary{\hbox{\char`\;}}{\Wrappedafterbreak}{\hbox{\char`\;}}}%
            \lccode`\~`\:\lowercase{\def~}{\discretionary{\hbox{\char`\:}}{\Wrappedafterbreak}{\hbox{\char`\:}}}%
            \lccode`\~`\?\lowercase{\def~}{\discretionary{\hbox{\char`\?}}{\Wrappedafterbreak}{\hbox{\char`\?}}}%
            \lccode`\~`\!\lowercase{\def~}{\discretionary{\hbox{\char`\!}}{\Wrappedafterbreak}{\hbox{\char`\!}}}%
            \lccode`\~`\/\lowercase{\def~}{\discretionary{\hbox{\char`\/}}{\Wrappedafterbreak}{\hbox{\char`\/}}}%
            \catcode`\.\active
            \catcode`\,\active
            \catcode`\;\active
            \catcode`\:\active
            \catcode`\?\active
            \catcode`\!\active
            \catcode`\/\active
            \lccode`\~`\~
        }
    \makeatother

    \let\OriginalVerbatim=\Verbatim
    \makeatletter
    \renewcommand{\Verbatim}[1][1]{%
        %\parskip\z@skip
        \sbox\Wrappedcontinuationbox {\Wrappedcontinuationsymbol}%
        \sbox\Wrappedvisiblespacebox {\FV@SetupFont\Wrappedvisiblespace}%
        \def\FancyVerbFormatLine ##1{\hsize\linewidth
            \vtop{\raggedright\hyphenpenalty\z@\exhyphenpenalty\z@
                \doublehyphendemerits\z@\finalhyphendemerits\z@
                \strut ##1\strut}%
        }%
        % If the linebreak is at a space, the latter will be displayed as visible
        % space at end of first line, and a continuation symbol starts next line.
        % Stretch/shrink are however usually zero for typewriter font.
        \def\FV@Space {%
            \nobreak\hskip\z@ plus\fontdimen3\font minus\fontdimen4\font
            \discretionary{\copy\Wrappedvisiblespacebox}{\Wrappedafterbreak}
            {\kern\fontdimen2\font}%
        }%

        % Allow breaks at special characters using \PYG... macros.
        \Wrappedbreaksatspecials
        % Breaks at punctuation characters . , ; ? ! and / need catcode=\active
        \OriginalVerbatim[#1,codes*=\Wrappedbreaksatpunct]%
    }
    \makeatother

    % Exact colors from NB
    \definecolor{incolor}{HTML}{303F9F}
    \definecolor{outcolor}{HTML}{D84315}
    \definecolor{cellborder}{HTML}{CFCFCF}
    \definecolor{cellbackground}{HTML}{F7F7F7}

    % prompt
    \makeatletter
    \newcommand{\boxspacing}{\kern\kvtcb@left@rule\kern\kvtcb@boxsep}
    \makeatother
    \newcommand{\prompt}[4]{
        {\ttfamily\llap{{\color{#2}[#3]:\hspace{3pt}#4}}\vspace{-\baselineskip}}
    }
    

    
    % Prevent overflowing lines due to hard-to-break entities
    \sloppy
    % Setup hyperref package
    \hypersetup{
      breaklinks=true,  % so long urls are correctly broken across lines
      colorlinks=true,
      urlcolor=urlcolor,
      linkcolor=linkcolor,
      citecolor=citecolor,
      }
    % Slightly bigger margins than the latex defaults
    
    \geometry{verbose,tmargin=1in,bmargin=1in,lmargin=1in,rmargin=1in}
    
    

\begin{document}
    
    \maketitle
    
    

    
    \hypertarget{primera-pregunta}{%
\section{Primera pregunta}\label{primera-pregunta}}

Sea \(\alpha\colon I \to \mathbb{R}^3\) una curva parametrizada y sea
\(v\in \mathbb{R}\) un vector fijo. Si \(\alpha'(t)\) es ortogonal a
\(v\) para todo \(t\in I\) y \(\alpha(0)\) también es ortogonal a \(v\),
demuestre que \(\alpha(t)\) es ortogonal a \(v\) para todo \(t\in I\).

\hypertarget{soluciuxf3n}{%
\subsection{Solución}\label{soluciuxf3n}}

    Sean \(\alpha_1,\alpha_2,\alpha_3\) funciones
\(\mathbb{R}\to\mathbb{R}\) tales que: \[
    \alpha(t) = (\alpha_1(t),\alpha_2(t),\alpha_3(t))
\] para todo \(t\in I\). Además, hagamos \(v = (v_1,v_2,v_3)\).

Entonces la derivada de \(\iprod{\alpha}{v}\) respecto de \(t\) viene
dada por: \begin{align*}
    \iprod{\alpha(t)}{v}' &= \alpha_1'(t)v_1 + \alpha_2'(t)v_2 + \alpha_3'(t)v_3\\
                          &= \iprod{\alpha'(t)}{v}\\
                          &= 0
\end{align*}

Ahora integrando respecto de \(t\) en un intervalo cualquiera
\((0,x)\in I\), obtenemos: \begin{align*}
    &\phantom{implies}
    \int_0^x \iprod{\alpha(t)}{v}' \;\mathrm{d}t = \int_0^x \alpha_1'(t)v_1 + \alpha_2'(t)v_2 + \alpha_3'(t)v_3 \;\mathrm{d}t = 0\\
    &\implies 
    v_1\int_0^x \alpha_1'(t) \;\mathrm{d}t + v_2\int_0^x \alpha_2'(t) \;\mathrm{d}t + v_3\int_0^x \alpha_3'(t) \;\mathrm{d}t = 0\\
    &\implies 
    v_1 (\alpha_1(x)-\alpha_1(0)) + v_2 (\alpha_2(x)-\alpha_2(0)) + v_3 (\alpha_3(x)-\alpha_3(0)) = 0 \\
    &\implies
    \iprod{\alpha(x)}{v} - \iprod{\alpha(0)}{v} = 0\\
    &\implies
    \iprod{\alpha(x)}{v} = 0.
\end{align*}

    \hypertarget{segunda-pregunta}{%
\section{Segunda Pregunta}\label{segunda-pregunta}}

\hypertarget{soluciuxf3n}{%
\subsection{Solución}\label{soluciuxf3n}}

\hypertarget{parametrizada-por-longitud-de-arco}{%
\subsubsection{Parametrizada por longitud de
arco}\label{parametrizada-por-longitud-de-arco}}

Basta con revisar que la norma de \(\alpha'\) es \(1\), pues en ese caso
\(s = L(t) = t-t_0\). Para los cálculos usaremos la librería sympy de
python.

    \begin{tcolorbox}[breakable, size=fbox, boxrule=1pt, pad at break*=1mm,colback=cellbackground, colframe=cellborder]
\prompt{In}{incolor}{1}{\boxspacing}
\begin{Verbatim}[commandchars=\\\{\}]
\PY{c+c1}{\PYZsh{} Import required libraries}
\PY{k+kn}{from} \PY{n+nn}{sympy} \PY{k+kn}{import} \PY{o}{*}
\PY{k+kn}{from} \PY{n+nn}{sympy}\PY{n+nn}{.}\PY{n+nn}{vector} \PY{k+kn}{import} \PY{o}{*}
\PY{k+kn}{from} \PY{n+nn}{sympy}\PY{n+nn}{.}\PY{n+nn}{plotting} \PY{k+kn}{import} \PY{o}{*}
\PY{k+kn}{from} \PY{n+nn}{matplotlib} \PY{k+kn}{import} \PY{o}{*}
\end{Verbatim}
\end{tcolorbox}

    Generemos unos gráficos para ver el trazo de \(\alpha\). Primero (en
negro) los valores \(a = b = 1\), luego (en azul) los valores
\(a=1/2, b=1\) y por último (en rojo) \(a = 1, b = 1/2\).

\emph{Nota:} en el código se usa \(r\) en vez de \(\alpha\) para hacer
más sencilla la escritura.

    \begin{tcolorbox}[breakable, size=fbox, boxrule=1pt, pad at break*=1mm,colback=cellbackground, colframe=cellborder]
\prompt{In}{incolor}{28}{\boxspacing}
\begin{Verbatim}[commandchars=\\\{\}]
\PY{n}{r}\PY{p}{,}\PY{n}{a}\PY{p}{,}\PY{n}{b}\PY{p}{,}\PY{n}{s}\PY{p}{,}\PY{n}{x}\PY{p}{,}\PY{n}{y}\PY{p}{,}\PY{n}{z} \PY{o}{=} \PY{n}{symbols}\PY{p}{(}\PY{l+s+s1}{\PYZsq{}}\PY{l+s+s1}{r,a,b,s,x,y,z}\PY{l+s+s1}{\PYZsq{}}\PY{p}{)}

\PY{c+c1}{\PYZsh{} Subtitute c = sqrt(a\PYZca{}2+b\PYZca{}2) to make computations simpler}
\PY{n}{x} \PY{o}{=} \PY{n}{a}\PY{o}{*}\PY{n}{cos}\PY{p}{(}\PY{n}{s}\PY{o}{/}\PY{n}{sqrt}\PY{p}{(}\PY{n}{a}\PY{o}{*}\PY{o}{*}\PY{l+m+mi}{2}\PY{o}{+}\PY{n}{b}\PY{o}{*}\PY{o}{*}\PY{l+m+mi}{2}\PY{p}{)}\PY{p}{)}
\PY{n}{y} \PY{o}{=} \PY{n}{a}\PY{o}{*}\PY{n}{sin}\PY{p}{(}\PY{n}{s}\PY{o}{/}\PY{n}{sqrt}\PY{p}{(}\PY{n}{a}\PY{o}{*}\PY{o}{*}\PY{l+m+mi}{2}\PY{o}{+}\PY{n}{b}\PY{o}{*}\PY{o}{*}\PY{l+m+mi}{2}\PY{p}{)}\PY{p}{)}
\PY{n}{z} \PY{o}{=} \PY{n}{b}\PY{o}{*}\PY{n}{s}\PY{o}{/}\PY{n}{sqrt}\PY{p}{(}\PY{n}{a}\PY{o}{*}\PY{o}{*}\PY{l+m+mi}{2}\PY{o}{+}\PY{n}{b}\PY{o}{*}\PY{o}{*}\PY{l+m+mi}{2}\PY{p}{)}

\PY{n}{rcParams}\PY{p}{[}\PY{l+s+s1}{\PYZsq{}}\PY{l+s+s1}{figure.figsize}\PY{l+s+s1}{\PYZsq{}}\PY{p}{]} \PY{o}{=} \PY{l+m+mi}{10}\PY{p}{,} \PY{l+m+mi}{10}

\PY{n}{p} \PY{o}{=} \PY{n}{plot3d\PYZus{}parametric\PYZus{}line}\PY{p}{(}
    \PY{p}{(}
        \PY{n}{x}\PY{o}{.}\PY{n}{subs}\PY{p}{(}\PY{p}{[}\PY{p}{(}\PY{n}{a}\PY{p}{,}\PY{l+m+mi}{1}\PY{p}{)}\PY{p}{,}\PY{p}{(}\PY{n}{b}\PY{p}{,}\PY{l+m+mi}{1}\PY{p}{)}\PY{p}{]}\PY{p}{)}\PY{p}{,}
        \PY{n}{y}\PY{o}{.}\PY{n}{subs}\PY{p}{(}\PY{p}{[}\PY{p}{(}\PY{n}{a}\PY{p}{,}\PY{l+m+mi}{1}\PY{p}{)}\PY{p}{,}\PY{p}{(}\PY{n}{b}\PY{p}{,}\PY{l+m+mi}{1}\PY{p}{)}\PY{p}{]}\PY{p}{)}\PY{p}{,}
        \PY{n}{z}\PY{o}{.}\PY{n}{subs}\PY{p}{(}\PY{p}{[}\PY{p}{(}\PY{n}{a}\PY{p}{,}\PY{l+m+mi}{1}\PY{p}{)}\PY{p}{,}\PY{p}{(}\PY{n}{b}\PY{p}{,}\PY{l+m+mi}{1}\PY{p}{)}\PY{p}{]}\PY{p}{)}\PY{p}{,}
        \PY{p}{(}\PY{n}{s}\PY{p}{,}\PY{l+m+mi}{0}\PY{p}{,}\PY{l+m+mi}{6}\PY{o}{*}\PY{n}{pi}\PY{p}{)}
    \PY{p}{)}\PY{p}{,}
    \PY{p}{(}
        \PY{n}{x}\PY{o}{.}\PY{n}{subs}\PY{p}{(}\PY{p}{[}\PY{p}{(}\PY{n}{a}\PY{p}{,}\PY{l+m+mi}{1}\PY{o}{/}\PY{l+m+mi}{2}\PY{p}{)}\PY{p}{,}\PY{p}{(}\PY{n}{b}\PY{p}{,}\PY{l+m+mi}{1}\PY{p}{)}\PY{p}{]}\PY{p}{)}\PY{p}{,}
        \PY{n}{y}\PY{o}{.}\PY{n}{subs}\PY{p}{(}\PY{p}{[}\PY{p}{(}\PY{n}{a}\PY{p}{,}\PY{l+m+mi}{1}\PY{o}{/}\PY{l+m+mi}{2}\PY{p}{)}\PY{p}{,}\PY{p}{(}\PY{n}{b}\PY{p}{,}\PY{l+m+mi}{1}\PY{p}{)}\PY{p}{]}\PY{p}{)}\PY{p}{,}
        \PY{n}{z}\PY{o}{.}\PY{n}{subs}\PY{p}{(}\PY{p}{[}\PY{p}{(}\PY{n}{a}\PY{p}{,}\PY{l+m+mi}{1}\PY{o}{/}\PY{l+m+mi}{2}\PY{p}{)}\PY{p}{,}\PY{p}{(}\PY{n}{b}\PY{p}{,}\PY{l+m+mi}{1}\PY{p}{)}\PY{p}{]}\PY{p}{)}\PY{p}{,}
        \PY{p}{(}\PY{n}{s}\PY{p}{,}\PY{l+m+mi}{0}\PY{p}{,}\PY{l+m+mi}{6}\PY{o}{*}\PY{n}{pi}\PY{p}{)}\PY{p}{,}
    \PY{p}{)}\PY{p}{,}
    \PY{p}{(}
        \PY{n}{x}\PY{o}{.}\PY{n}{subs}\PY{p}{(}\PY{p}{[}\PY{p}{(}\PY{n}{a}\PY{p}{,}\PY{l+m+mi}{1}\PY{p}{)}\PY{p}{,}\PY{p}{(}\PY{n}{b}\PY{p}{,}\PY{l+m+mi}{1}\PY{o}{/}\PY{l+m+mi}{2}\PY{p}{)}\PY{p}{]}\PY{p}{)}\PY{p}{,}
        \PY{n}{y}\PY{o}{.}\PY{n}{subs}\PY{p}{(}\PY{p}{[}\PY{p}{(}\PY{n}{a}\PY{p}{,}\PY{l+m+mi}{1}\PY{p}{)}\PY{p}{,}\PY{p}{(}\PY{n}{b}\PY{p}{,}\PY{l+m+mi}{1}\PY{o}{/}\PY{l+m+mi}{2}\PY{p}{)}\PY{p}{]}\PY{p}{)}\PY{p}{,}
        \PY{n}{z}\PY{o}{.}\PY{n}{subs}\PY{p}{(}\PY{p}{[}\PY{p}{(}\PY{n}{a}\PY{p}{,}\PY{l+m+mi}{1}\PY{p}{)}\PY{p}{,}\PY{p}{(}\PY{n}{b}\PY{p}{,}\PY{l+m+mi}{1}\PY{o}{/}\PY{l+m+mi}{2}\PY{p}{)}\PY{p}{]}\PY{p}{)}\PY{p}{,}
        \PY{p}{(}\PY{n}{s}\PY{p}{,}\PY{l+m+mi}{0}\PY{p}{,}\PY{l+m+mi}{8}\PY{o}{*}\PY{n}{pi}\PY{p}{)}\PY{p}{,}
    \PY{p}{)}\PY{p}{,}
    \PY{n}{show}\PY{o}{=}\PY{k+kc}{False}\PY{p}{,}
    \PY{n}{title}\PY{o}{=}\PY{l+s+s1}{\PYZsq{}}\PY{l+s+s1}{Grafico de tres espirales}\PY{l+s+s1}{\PYZsq{}}\PY{p}{,}
\PY{p}{)}

\PY{n}{p}\PY{p}{[}\PY{l+m+mi}{0}\PY{p}{]}\PY{o}{.}\PY{n}{line\PYZus{}color}\PY{o}{=}\PY{l+s+s1}{\PYZsq{}}\PY{l+s+s1}{black}\PY{l+s+s1}{\PYZsq{}}
\PY{n}{p}\PY{p}{[}\PY{l+m+mi}{1}\PY{p}{]}\PY{o}{.}\PY{n}{line\PYZus{}color}\PY{o}{=}\PY{l+s+s1}{\PYZsq{}}\PY{l+s+s1}{blue}\PY{l+s+s1}{\PYZsq{}}
\PY{n}{p}\PY{p}{[}\PY{l+m+mi}{2}\PY{p}{]}\PY{o}{.}\PY{n}{line\PYZus{}color}\PY{o}{=}\PY{l+s+s1}{\PYZsq{}}\PY{l+s+s1}{red}\PY{l+s+s1}{\PYZsq{}}
\PY{n}{p}\PY{o}{.}\PY{n}{show}\PY{p}{(}\PY{p}{)}
\end{Verbatim}
\end{tcolorbox}

    \begin{center}
    \adjustimage{max size={0.9\linewidth}{0.9\paperheight}}{Tarea1_files/Tarea1_5_0.png}
    \end{center}
    { \hspace*{\fill} \\}
    
    Calculemos ahora \(\alpha'\).

    \begin{tcolorbox}[breakable, size=fbox, boxrule=1pt, pad at break*=1mm,colback=cellbackground, colframe=cellborder]
\prompt{In}{incolor}{6}{\boxspacing}
\begin{Verbatim}[commandchars=\\\{\}]
\PY{n}{r} \PY{o}{=} \PY{n}{Matrix}\PY{p}{(}\PY{p}{[}\PY{n}{x}\PY{p}{,}\PY{n}{y}\PY{p}{,}\PY{n}{z}\PY{p}{]}\PY{p}{)}

\PY{n}{drds} \PY{o}{=} \PY{n}{diff}\PY{p}{(}\PY{n}{r}\PY{p}{,}\PY{n}{s}\PY{p}{)}
\PY{n}{drds}
\end{Verbatim}
\end{tcolorbox}
 
            
\prompt{Out}{outcolor}{6}{}
    
    $\displaystyle \left[\begin{matrix}- \frac{a \sin{\left(\frac{s}{\sqrt{a^{2} + b^{2}}} \right)}}{\sqrt{a^{2} + b^{2}}}\\\frac{a \cos{\left(\frac{s}{\sqrt{a^{2} + b^{2}}} \right)}}{\sqrt{a^{2} + b^{2}}}\\\frac{b}{\sqrt{a^{2} + b^{2}}}\end{matrix}\right]$

    

    Podemos obtener su norma y simplificar.

    \begin{tcolorbox}[breakable, size=fbox, boxrule=1pt, pad at break*=1mm,colback=cellbackground, colframe=cellborder]
\prompt{In}{incolor}{7}{\boxspacing}
\begin{Verbatim}[commandchars=\\\{\}]
\PY{n}{normdrds} \PY{o}{=} \PY{n}{sqrt}\PY{p}{(}\PY{n}{drds}\PY{o}{.}\PY{n}{dot}\PY{p}{(}\PY{n}{drds}\PY{p}{)}\PY{p}{)}
\PY{n}{normdrds}
\end{Verbatim}
\end{tcolorbox}
 
            
\prompt{Out}{outcolor}{7}{}
    
    $\displaystyle \sqrt{\frac{a^{2} \sin^{2}{\left(\frac{s}{\sqrt{a^{2} + b^{2}}} \right)}}{a^{2} + b^{2}} + \frac{a^{2} \cos^{2}{\left(\frac{s}{\sqrt{a^{2} + b^{2}}} \right)}}{a^{2} + b^{2}} + \frac{b^{2}}{a^{2} + b^{2}}}$

    

    \begin{tcolorbox}[breakable, size=fbox, boxrule=1pt, pad at break*=1mm,colback=cellbackground, colframe=cellborder]
\prompt{In}{incolor}{8}{\boxspacing}
\begin{Verbatim}[commandchars=\\\{\}]
\PY{n}{simplify}\PY{p}{(}\PY{n}{normdrds}\PY{p}{)}
\end{Verbatim}
\end{tcolorbox}
 
            
\prompt{Out}{outcolor}{8}{}
    
    $\displaystyle 1$

    

    Con lo que la norma es \(1\), que es lo que queríamos demostrar.

    \hypertarget{curvatura-y-torsiuxf3n}{%
\subsubsection{Curvatura y torsión}\label{curvatura-y-torsiuxf3n}}

La curvatura es la norma de la derivada segunda de \(\alpha\) con
respecto de \(s\). Primero calculamos \(\alpha''\) y luego su norma:

    \begin{tcolorbox}[breakable, size=fbox, boxrule=1pt, pad at break*=1mm,colback=cellbackground, colframe=cellborder]
\prompt{In}{incolor}{9}{\boxspacing}
\begin{Verbatim}[commandchars=\\\{\}]
\PY{n}{ddrdss} \PY{o}{=} \PY{n}{diff}\PY{p}{(}\PY{n}{drds}\PY{p}{,}\PY{n}{s}\PY{p}{)}
\PY{n}{ddrdss}
\end{Verbatim}
\end{tcolorbox}
 
            
\prompt{Out}{outcolor}{9}{}
    
    $\displaystyle \left[\begin{matrix}- \frac{a \cos{\left(\frac{s}{\sqrt{a^{2} + b^{2}}} \right)}}{a^{2} + b^{2}}\\- \frac{a \sin{\left(\frac{s}{\sqrt{a^{2} + b^{2}}} \right)}}{a^{2} + b^{2}}\\0\end{matrix}\right]$

    

    \begin{tcolorbox}[breakable, size=fbox, boxrule=1pt, pad at break*=1mm,colback=cellbackground, colframe=cellborder]
\prompt{In}{incolor}{10}{\boxspacing}
\begin{Verbatim}[commandchars=\\\{\}]
\PY{n}{normddrdss} \PY{o}{=} \PY{n}{sqrt}\PY{p}{(}\PY{n}{ddrdss}\PY{o}{.}\PY{n}{dot}\PY{p}{(}\PY{n}{ddrdss}\PY{p}{)}\PY{p}{)}
\PY{n}{normddrdss}
\end{Verbatim}
\end{tcolorbox}
 
            
\prompt{Out}{outcolor}{10}{}
    
    $\displaystyle \sqrt{\frac{a^{2} \sin^{2}{\left(\frac{s}{\sqrt{a^{2} + b^{2}}} \right)}}{\left(a^{2} + b^{2}\right)^{2}} + \frac{a^{2} \cos^{2}{\left(\frac{s}{\sqrt{a^{2} + b^{2}}} \right)}}{\left(a^{2} + b^{2}\right)^{2}}}$

    

    \begin{tcolorbox}[breakable, size=fbox, boxrule=1pt, pad at break*=1mm,colback=cellbackground, colframe=cellborder]
\prompt{In}{incolor}{11}{\boxspacing}
\begin{Verbatim}[commandchars=\\\{\}]
\PY{n}{simplify}\PY{p}{(}\PY{n}{normddrdss}\PY{p}{)}
\end{Verbatim}
\end{tcolorbox}
 
            
\prompt{Out}{outcolor}{11}{}
    
    $\displaystyle \sqrt{\frac{a^{2}}{\left(a^{2} + b^{2}\right)^{2}}}$

    

    La expresión anterior nos da la curvatura de \(\alpha\).

    La torsión es el múltiplo escalar por el que diferen el vector normal y
la primera derivada del vector binormal. Podemos comparar ambos vectores
para conseguir ese múltiplo. Calculamos primero el vector normal:

    \begin{tcolorbox}[breakable, size=fbox, boxrule=1pt, pad at break*=1mm,colback=cellbackground, colframe=cellborder]
\prompt{In}{incolor}{12}{\boxspacing}
\begin{Verbatim}[commandchars=\\\{\}]
\PY{n}{normddrdss} \PY{o}{=} \PY{n}{simplify}\PY{p}{(}\PY{n}{normddrdss}\PY{p}{)} \PY{c+c1}{\PYZsh{} Take the second derivative to be its simplified form}
\PY{n}{n} \PY{o}{=} \PY{n}{ddrdss}\PY{o}{/}\PY{n}{normddrdss}
\PY{n}{n}
\end{Verbatim}
\end{tcolorbox}
 
            
\prompt{Out}{outcolor}{12}{}
    
    $\displaystyle \left[\begin{matrix}- \frac{a \cos{\left(\frac{s}{\sqrt{a^{2} + b^{2}}} \right)}}{\sqrt{\frac{a^{2}}{\left(a^{2} + b^{2}\right)^{2}}} \left(a^{2} + b^{2}\right)}\\- \frac{a \sin{\left(\frac{s}{\sqrt{a^{2} + b^{2}}} \right)}}{\sqrt{\frac{a^{2}}{\left(a^{2} + b^{2}\right)^{2}}} \left(a^{2} + b^{2}\right)}\\0\end{matrix}\right]$

    

    Calculamos ahora el vector tangente usando la primera derivada:

    \begin{tcolorbox}[breakable, size=fbox, boxrule=1pt, pad at break*=1mm,colback=cellbackground, colframe=cellborder]
\prompt{In}{incolor}{13}{\boxspacing}
\begin{Verbatim}[commandchars=\\\{\}]
\PY{n}{normdrds} \PY{o}{=} \PY{n}{simplify}\PY{p}{(}\PY{n}{normdrds}\PY{p}{)}
\PY{n}{t} \PY{o}{=} \PY{n}{drds}\PY{o}{/}\PY{n}{normdrds}
\PY{n}{t}
\end{Verbatim}
\end{tcolorbox}
 
            
\prompt{Out}{outcolor}{13}{}
    
    $\displaystyle \left[\begin{matrix}- \frac{a \sin{\left(\frac{s}{\sqrt{a^{2} + b^{2}}} \right)}}{\sqrt{a^{2} + b^{2}}}\\\frac{a \cos{\left(\frac{s}{\sqrt{a^{2} + b^{2}}} \right)}}{\sqrt{a^{2} + b^{2}}}\\\frac{b}{\sqrt{a^{2} + b^{2}}}\end{matrix}\right]$

    

    Podemos entonces calcular el vector binormal y simplificarlo:

    \begin{tcolorbox}[breakable, size=fbox, boxrule=1pt, pad at break*=1mm,colback=cellbackground, colframe=cellborder]
\prompt{In}{incolor}{14}{\boxspacing}
\begin{Verbatim}[commandchars=\\\{\}]
\PY{n}{bi} \PY{o}{=} \PY{n}{t}\PY{o}{.}\PY{n}{cross}\PY{p}{(}\PY{n}{n}\PY{p}{)}
\PY{n}{bi}
\end{Verbatim}
\end{tcolorbox}
 
            
\prompt{Out}{outcolor}{14}{}
    
    $\displaystyle \left[\begin{matrix}\frac{a b \sin{\left(\frac{s}{\sqrt{a^{2} + b^{2}}} \right)}}{\sqrt{\frac{a^{2}}{\left(a^{2} + b^{2}\right)^{2}}} \left(a^{2} + b^{2}\right)^{\frac{3}{2}}}\\- \frac{a b \cos{\left(\frac{s}{\sqrt{a^{2} + b^{2}}} \right)}}{\sqrt{\frac{a^{2}}{\left(a^{2} + b^{2}\right)^{2}}} \left(a^{2} + b^{2}\right)^{\frac{3}{2}}}\\\frac{a^{2} \sin^{2}{\left(\frac{s}{\sqrt{a^{2} + b^{2}}} \right)}}{\sqrt{\frac{a^{2}}{\left(a^{2} + b^{2}\right)^{2}}} \left(a^{2} + b^{2}\right)^{\frac{3}{2}}} + \frac{a^{2} \cos^{2}{\left(\frac{s}{\sqrt{a^{2} + b^{2}}} \right)}}{\sqrt{\frac{a^{2}}{\left(a^{2} + b^{2}\right)^{2}}} \left(a^{2} + b^{2}\right)^{\frac{3}{2}}}\end{matrix}\right]$

    

    \begin{tcolorbox}[breakable, size=fbox, boxrule=1pt, pad at break*=1mm,colback=cellbackground, colframe=cellborder]
\prompt{In}{incolor}{15}{\boxspacing}
\begin{Verbatim}[commandchars=\\\{\}]
\PY{n}{simplify}\PY{p}{(}\PY{n}{bi}\PY{p}{)}
\end{Verbatim}
\end{tcolorbox}
 
            
\prompt{Out}{outcolor}{15}{}
    
    $\displaystyle \left[\begin{matrix}\frac{a b \sin{\left(\frac{s}{\sqrt{a^{2} + b^{2}}} \right)}}{\sqrt{\frac{a^{2}}{\left(a^{2} + b^{2}\right)^{2}}} \left(a^{2} + b^{2}\right)^{\frac{3}{2}}}\\- \frac{a b \cos{\left(\frac{s}{\sqrt{a^{2} + b^{2}}} \right)}}{\sqrt{\frac{a^{2}}{\left(a^{2} + b^{2}\right)^{2}}} \left(a^{2} + b^{2}\right)^{\frac{3}{2}}}\\\frac{a^{2}}{\sqrt{\frac{a^{2}}{\left(a^{2} + b^{2}\right)^{2}}} \left(a^{2} + b^{2}\right)^{\frac{3}{2}}}\end{matrix}\right]$

    

    Solo falta la derivada del vector binormal:

    \begin{tcolorbox}[breakable, size=fbox, boxrule=1pt, pad at break*=1mm,colback=cellbackground, colframe=cellborder]
\prompt{In}{incolor}{16}{\boxspacing}
\begin{Verbatim}[commandchars=\\\{\}]
\PY{n}{bi} \PY{o}{=} \PY{n}{simplify}\PY{p}{(}\PY{n}{bi}\PY{p}{)}
\PY{n}{dbids} \PY{o}{=} \PY{n}{diff}\PY{p}{(}\PY{n}{bi}\PY{p}{,}\PY{n}{s}\PY{p}{)}
\PY{n}{dbids}
\end{Verbatim}
\end{tcolorbox}
 
            
\prompt{Out}{outcolor}{16}{}
    
    $\displaystyle \left[\begin{matrix}\frac{a b \cos{\left(\frac{s}{\sqrt{a^{2} + b^{2}}} \right)}}{\sqrt{\frac{a^{2}}{\left(a^{2} + b^{2}\right)^{2}}} \left(a^{2} + b^{2}\right)^{2}}\\\frac{a b \sin{\left(\frac{s}{\sqrt{a^{2} + b^{2}}} \right)}}{\sqrt{\frac{a^{2}}{\left(a^{2} + b^{2}\right)^{2}}} \left(a^{2} + b^{2}\right)^{2}}\\0\end{matrix}\right]$

    

    Podemos entonces comparar \(b'\) con \(n\):

    \begin{tcolorbox}[breakable, size=fbox, boxrule=1pt, pad at break*=1mm,colback=cellbackground, colframe=cellborder]
\prompt{In}{incolor}{17}{\boxspacing}
\begin{Verbatim}[commandchars=\\\{\}]
\PY{n}{display}\PY{p}{(}\PY{n}{Eq}\PY{p}{(}\PY{n}{dbids}\PY{p}{,}\PY{n}{n}\PY{p}{)}\PY{p}{)}
\end{Verbatim}
\end{tcolorbox}

    $\displaystyle \left[\begin{matrix}\frac{a b \cos{\left(\frac{s}{\sqrt{a^{2} + b^{2}}} \right)}}{\sqrt{\frac{a^{2}}{\left(a^{2} + b^{2}\right)^{2}}} \left(a^{2} + b^{2}\right)^{2}}\\\frac{a b \sin{\left(\frac{s}{\sqrt{a^{2} + b^{2}}} \right)}}{\sqrt{\frac{a^{2}}{\left(a^{2} + b^{2}\right)^{2}}} \left(a^{2} + b^{2}\right)^{2}}\\0\end{matrix}\right] = \left[\begin{matrix}- \frac{a \cos{\left(\frac{s}{\sqrt{a^{2} + b^{2}}} \right)}}{\sqrt{\frac{a^{2}}{\left(a^{2} + b^{2}\right)^{2}}} \left(a^{2} + b^{2}\right)}\\- \frac{a \sin{\left(\frac{s}{\sqrt{a^{2} + b^{2}}} \right)}}{\sqrt{\frac{a^{2}}{\left(a^{2} + b^{2}\right)^{2}}} \left(a^{2} + b^{2}\right)}\\0\end{matrix}\right]$

    
    De donde podemos ver que la torsión es: \[
    \frac{b}{a^2+b^2}
\]

    \hypertarget{plano-osculante}{%
\subsubsection{Plano osculante}\label{plano-osculante}}

En la ecuación del plano en el punto \(s = (s_1,s_2,s_3)\): \[
    a (x-s_1) + b (y-s_2) + c(z-s_3),
\] los coeficientes \(a,b,c\) coinciden con los del vector normal:

    \begin{tcolorbox}[breakable, size=fbox, boxrule=1pt, pad at break*=1mm,colback=cellbackground, colframe=cellborder]
\prompt{In}{incolor}{18}{\boxspacing}
\begin{Verbatim}[commandchars=\\\{\}]
\PY{n}{n}
\end{Verbatim}
\end{tcolorbox}
 
            
\prompt{Out}{outcolor}{18}{}
    
    $\displaystyle \left[\begin{matrix}- \frac{a \cos{\left(\frac{s}{\sqrt{a^{2} + b^{2}}} \right)}}{\sqrt{\frac{a^{2}}{\left(a^{2} + b^{2}\right)^{2}}} \left(a^{2} + b^{2}\right)}\\- \frac{a \sin{\left(\frac{s}{\sqrt{a^{2} + b^{2}}} \right)}}{\sqrt{\frac{a^{2}}{\left(a^{2} + b^{2}\right)^{2}}} \left(a^{2} + b^{2}\right)}\\0\end{matrix}\right]$

    

    \hypertarget{tercera-pregunta}{%
\subsection{Tercera pregunta}\label{tercera-pregunta}}

    Veamos primero el trazo de la curva \(\alpha\).

    \begin{tcolorbox}[breakable, size=fbox, boxrule=1pt, pad at break*=1mm,colback=cellbackground, colframe=cellborder]
\prompt{In}{incolor}{41}{\boxspacing}
\begin{Verbatim}[commandchars=\\\{\}]
\PY{n}{t} \PY{o}{=} \PY{n}{symbols}\PY{p}{(}\PY{l+s+s1}{\PYZsq{}}\PY{l+s+s1}{t}\PY{l+s+s1}{\PYZsq{}}\PY{p}{)}

\PY{n}{x} \PY{o}{=} \PY{n}{t}
\PY{n}{y} \PY{o}{=} \PY{p}{(}\PY{n}{exp}\PY{p}{(}\PY{n}{t}\PY{p}{)}\PY{o}{+}\PY{n}{exp}\PY{p}{(}\PY{o}{\PYZhy{}}\PY{n}{t}\PY{p}{)}\PY{p}{)}\PY{o}{/}\PY{l+m+mi}{2}

\PY{n}{rcParams}\PY{p}{[}\PY{l+s+s1}{\PYZsq{}}\PY{l+s+s1}{figure.figsize}\PY{l+s+s1}{\PYZsq{}}\PY{p}{]} \PY{o}{=} \PY{l+m+mi}{8}\PY{p}{,} \PY{l+m+mi}{8}

\PY{n}{plot\PYZus{}parametric}\PY{p}{(}
    \PY{p}{(}\PY{n}{x}\PY{p}{,} \PY{n}{y}\PY{p}{,} \PY{p}{(}\PY{n}{t}\PY{p}{,} \PY{o}{\PYZhy{}}\PY{l+m+mi}{1}\PY{p}{,} \PY{l+m+mi}{1}\PY{p}{)}\PY{p}{)}\PY{p}{,}
\PY{p}{)}
\end{Verbatim}
\end{tcolorbox}

    \begin{center}
    \adjustimage{max size={0.9\linewidth}{0.9\paperheight}}{Tarea1_files/Tarea1_33_0.png}
    \end{center}
    { \hspace*{\fill} \\}
    
            \begin{tcolorbox}[breakable, size=fbox, boxrule=.5pt, pad at break*=1mm, opacityfill=0]
\prompt{Out}{outcolor}{41}{\boxspacing}
\begin{Verbatim}[commandchars=\\\{\}]
<sympy.plotting.plot.Plot at 0x7f02444de680>
\end{Verbatim}
\end{tcolorbox}
        
    \hypertarget{parametrizaciuxf3n-por-longitud-de-arco}{%
\subsubsection{Parametrización por longitud de
arco}\label{parametrizaciuxf3n-por-longitud-de-arco}}

Calculamos primero \(\alpha'\):

    \begin{tcolorbox}[breakable, size=fbox, boxrule=1pt, pad at break*=1mm,colback=cellbackground, colframe=cellborder]
\prompt{In}{incolor}{51}{\boxspacing}
\begin{Verbatim}[commandchars=\\\{\}]
\PY{n}{dxdt} \PY{o}{=} \PY{n}{diff}\PY{p}{(}\PY{n}{x}\PY{p}{,}\PY{n}{t}\PY{p}{)}
\PY{n}{dydt} \PY{o}{=} \PY{n}{diff}\PY{p}{(}\PY{n}{y}\PY{p}{,}\PY{n}{t}\PY{p}{)}
\PY{n}{display}\PY{p}{(}\PY{n}{dxds}\PY{p}{,}\PY{n}{dyds}\PY{p}{)}
\end{Verbatim}
\end{tcolorbox}

    $\displaystyle 1$

    
    $\displaystyle \frac{e^{t}}{2} - \frac{e^{- t}}{2}$

    
    Luego, su norma:

    \begin{tcolorbox}[breakable, size=fbox, boxrule=1pt, pad at break*=1mm,colback=cellbackground, colframe=cellborder]
\prompt{In}{incolor}{52}{\boxspacing}
\begin{Verbatim}[commandchars=\\\{\}]
\PY{n}{normdrdt} \PY{o}{=} \PY{n}{sqrt}\PY{p}{(}\PY{n}{dxdt}\PY{o}{*}\PY{o}{*}\PY{l+m+mi}{2} \PY{o}{+} \PY{n}{dydt}\PY{o}{*}\PY{o}{*}\PY{l+m+mi}{2}\PY{p}{)}
\PY{n}{normdrdt}
\end{Verbatim}
\end{tcolorbox}
 
            
\prompt{Out}{outcolor}{52}{}
    
    $\displaystyle \sqrt{\left(\frac{e^{t}}{2} - \frac{e^{- t}}{2}\right)^{2} + 1}$

    

    \begin{tcolorbox}[breakable, size=fbox, boxrule=1pt, pad at break*=1mm,colback=cellbackground, colframe=cellborder]
\prompt{In}{incolor}{44}{\boxspacing}
\begin{Verbatim}[commandchars=\\\{\}]
\PY{n}{simplify}\PY{p}{(}\PY{n}{normdrdt}\PY{p}{)}
\end{Verbatim}
\end{tcolorbox}
 
            
\prompt{Out}{outcolor}{44}{}
    
    $\displaystyle \frac{\sqrt{2 \cosh{\left(2 t \right)} + 2}}{2}$

    

    \begin{tcolorbox}[breakable, size=fbox, boxrule=1pt, pad at break*=1mm,colback=cellbackground, colframe=cellborder]
\prompt{In}{incolor}{56}{\boxspacing}
\begin{Verbatim}[commandchars=\\\{\}]
\PY{n}{integrate}\PY{p}{(}\PY{n}{sqrt}\PY{p}{(}\PY{n}{dxdt}\PY{o}{*}\PY{o}{*}\PY{l+m+mi}{2} \PY{o}{+} \PY{n}{dydt}\PY{o}{*}\PY{o}{*}\PY{l+m+mi}{2}\PY{p}{)}\PY{p}{,} \PY{n}{t}\PY{p}{)}
\end{Verbatim}
\end{tcolorbox}
 
            
\prompt{Out}{outcolor}{56}{}
    
    $\displaystyle \frac{\int \sqrt{e^{2 t} + 2 + e^{- 2 t}}\, dt}{2}$

    

    \begin{tcolorbox}[breakable, size=fbox, boxrule=1pt, pad at break*=1mm,colback=cellbackground, colframe=cellborder]
\prompt{In}{incolor}{ }{\boxspacing}
\begin{Verbatim}[commandchars=\\\{\}]

\end{Verbatim}
\end{tcolorbox}


    % Add a bibliography block to the postdoc
    
    
    
\end{document}
